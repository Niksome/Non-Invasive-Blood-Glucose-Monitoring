% Version 1.2.0 DE
%
% Seitenbegrenzungen anzeigen
% \usepackage{showframe}

%%%%%%%%%%%%%%%%%%%%%%%%%%%%%%%%%%%%%%%%%%%%%%%%%%
%%% 		Manuelle Zeilenumbrüche!		  %%%%
%%% 		im Abbildungsverzeichnis!		  %%%%
%%% 		und Tabellenverzeichnis!		  %%%%
%%%%%%%%%%%%%%%%%%%%%%%%%%%%%%%%%%%%%%%%%%%%%%%%%%

% Zeilenumbrüche manuell in \listoffigures und \listoftables setzen mit "\protect\\" (ohne Anführungszeichen) im optionalen Parameter des "\caption"-Befehls
% Beispiel: 
%\caption[Exterior view of the KIT library. Consetetur sadipscing elitr,\protect\\ sed diam nonumy eirmod tempor invidunt ut labore]{Exterior view of the KIT library. Consetetur sadipscing elitr, sed diam nonumy eirmod tempor invidunt ut labore}

%%%%%%%%%%%%%%%%%%%%%%%%%%%%%%%%%%%%%%%%%%%%%%%%%%
%%% 				Sonderzeichen!			  %%%%
%%%%%%%%%%%%%%%%%%%%%%%%%%%%%%%%%%%%%%%%%%%%%%%%%%

% Geschütztes Leerzeichen: ~ 
% Geschütztes schmales Leerzeichen: \,
% Backslash: \textbackslash
% Prozentzeichen: \%
% Et-Zeichen: \&
% Zirkumflex: \^
% Tilde: \~
% Dollar: \$
% Raute: \#
% Unterstrich: \_
% Geschweifte Klammern: \{
% Geschützter Trennstrich (Divis): "~	% Erfordert das Paket "\usepackage[ngerman]{babel}"
% Geschützter Trennstrich (Divis) Alternative: \hbox{-}
% Bedingter Trennsricht (Divis): \-

%%%%%%%%%%%%%%%%%%%%%%%%%%%%%%%%%%%%%%%%%%%%%%%%%%
%%% 	Allgemeine Einstellungen!			  %%%%
%%%%%%%%%%%%%%%%%%%%%%%%%%%%%%%%%%%%%%%%%%%%%%%%%%

% Seitenränder einstellen
\setlength{\topskip}{10.5pt} % Verhindern einer Fehlermeldung (Zusatz: siehe Kohm 2020: S. 36, Tabelle 2.1 "Satzspiegelmaße in Abhängigkeit von DIV bei A4 ohne Berücksichtigung von \topskip oder BCOR)

%%%%%%%%%%%%%%%%%%%%%%%%%%%%%%%%%%%%%%%%%%%%%%%%%%
%%%        		Skalierung!					  %%%%
%%%%%%%%%%%%%%%%%%%%%%%%%%%%%%%%%%%%%%%%%%%%%%%%%%

% Bitte skalieren Sie, entsprechend der Seitenanzahl ihres Dokuments, die Seitenränder wie folgt (siehe optionale Parameter im folgenden Paket "\usepackage{geometry}":
% - bis 199 Seiten ("inner": 20mm, "outer": 15-18mm) >>> textwidth=113mm
% - 200 bis 399 Seiten ("inner": 23mm, "outer": 15-18mm) >>> textwidth=110mm
% - ab 400 Seiten ("inner": 25mm, "outer": 15mm) >>> textwidth=108mm

%%%%%%%%%%%%%%%%%%%%%%%%%%%%%%%%%%%%%%%%%%%%%%%%%%
%%%        		Papierformat A5!			   %%%
%%%%%%%%%%%%%%%%%%%%%%%%%%%%%%%%%%%%%%%%%%%%%%%%%%

% Hinweis: Die Seitenränder für die Skalierung (s. o. "Skalierung") kann man mithilfe des Pakets "geometry" verwalten, siehe bitte hierzu die entsprechende Dokumentation unter: http://ftp.fau.de/ctan/macros/latex/contrib/geometry/geometry.pdf

% Geometry!
\usepackage[%
papersize={17cm,24cm},% Papierformat. Andere: "papersize={17cm,24cm}", "a5paper"
headheight=1.5\baselineskip,% Linie unterhalb der Kopfzeile
top=25mm,% Abstand vom oberen Rand
inner=20mm,% Abstand vom inneren Rand
outer=15mm,% Abstand vom äußeren Rand
%lines=38,% Anzahl der Linien. Auskommentierung empfohlen, da es zu Problemen bei der korrekten Berechnung führt
%textwidth=113mm,% Breite des Satzspiegels. Auskommentierung empfohlen, da es zu Problemen bei der korrekten Berechnung führt
footnotesep=7mm,% Vertikaler Abstand zwischen Satzspiegel und Fußnotenlinie
heightrounded=true%
]{geometry}

%%%%%%%%%%%%%%%%%%%%%%%%%%%%%%%%%%%%%%%%%%%%%%%%%%
%%%        			Fußzeile!			   	   %%%
%%%%%%%%%%%%%%%%%%%%%%%%%%%%%%%%%%%%%%%%%%%%%%%%%%

% Abstand zwischen Textkörper/Satzspiegel und Unterkante Fußzeile (Seitenzahlen)
\setlength{\footskip}{13.6mm}

% Abstand zwischen Fließtext und Fußnotentrennlinie 
\setlength{\skip\footins}{20pt}

%%%%%%%%%%%%%%%%%%%%%%%%%%%%%%%%%%%%%%%%%%%%%%%%%%
%%%        		Fußnote!				   	   %%%
%%%%%%%%%%%%%%%%%%%%%%%%%%%%%%%%%%%%%%%%%%%%%%%%%%

% (1) Position und Größe der Fußnoten bei 2-stelligen Fußnotennummern
\deffootnote[1.8em]{1.8em}{0em}{\makebox[1.7em][l]{\textsuperscript{\thefootnotemark\ }}}

% (2) Bitte wählen Sie für 3-stellige Fußnotennummern den folgenden Befehl und deaktivieren Sie (1), indem Sie den vorhergehenden Befehl "\deffootnote{1.7em}{0em}{\makebox[1.7em][l]{\thefootnotemark}}" mit % auskommentieren
%\deffootnote{2.2em}{0em}{\makebox[2.2em][l]{\thefootnotemark}}

% Verhindert das Fortsetzen von Fussnoten auf der gegenüberliegenden Seite
\interfootnotelinepenalty=10000 

%%%%%%%%%%%%%%%%%%%%%%%%%%%%%%%%%%%%%%%%%%%%%%%%%%
%%%  				Absatz!				  	   %%%
%%%%%%%%%%%%%%%%%%%%%%%%%%%%%%%%%%%%%%%%%%%%%%%%%%

% Zeilen auf der Seite verteilen (Es wird kein Ausgleich des unteren Seitenrandes durch Dehnung der Absatzabstände durchgeführt) 
\raggedbottom   

% Einzugtiefe (horizontaler Abstand) der ersten Zeile des Absatzes
\setlength{\parindent}{0pt}

% Vertikaler Abstand zwischen den Absätzen
\setlength{\parskip}{2.5mm}

% Änderung der Abstände zwischen den Einträgen
% \setlength{\parskip}{0.3\baselineskip plus 0.15\baselineskip minus 0.15\baselineskip} 

% Schusterjungen (einzelne Zeile unten auf der Seite) unterdrücken
\clubpenalty = 10000 

% Hurenkinder (einzelne Zeile oben auf der Seite) unterdrücken
\widowpenalty = 10000
\displaywidowpenalty = 10000

% Silbentrennung am Seitenumbruch verhindern
\brokenpenalty = 10000

%%%%%%%%%%%%%%%%%%%%%%%%%%%%%%%%%%%%%%%%%%%%%%%%%%
%%%  		Caption! Beschriftungen!		   %%%
%%%			Abbildungen! (figure) 			   %%%
%%%			Tabellen! (table)			   	   %%%
%%%%%%%%%%%%%%%%%%%%%%%%%%%%%%%%%%%%%%%%%%%%%%%%%%

\usepackage[%
labelfont=bf,% Fette Beschriftungen für die Bezeichnung "Abbildung" und "Tabelle"
font=footnotesize% Schriftgröße für Beschriftungen
]{caption}

% Abbildungen
\captionsetup[figure]{position=below} 
\captionsetup[figure]{aboveskip=3mm} %  Abstand Bild-Bildunterschrift
\captionsetup[figure]{belowskip=-2mm} % Abstand Bildunterschrift-Fließtext

% Tabellen
% Bei der Nutzung von "\captionabove" werden die Werte "aboveskip" und "belowskip" vertauscht; bitte nutzen Sie daher den Befehl "\caption" für Tabellenüberschriften
\captionsetup[table]{position=top} 
\captionsetup[table]{aboveskip=0.8mm} % Abstand: Fließtext-Tabellenüberschrift (im PDF gemessen=~7mm) | Anmerkung: mit dem Befehl "\captionabove" ändert sich die Reihenfolge der Tabellenüberschrift  in: Tabellenüberschrift-Tabelle
% \captionsetup[table]{aboveskip=1.5mm} % Abstand: Fließtext-Tabellenüberschrift (im PDF gemessen=~7mm) | Anmerkung: mit dem Befehl "\captionabove" ändert sich die Reihenfolge der Tabellenüberschrift  in: Tabellenüberschrift-Tabelle
\captionsetup[table]{belowskip=2.5mm} % Abstand: Tabellenüberschrift-Tabelle (im PDF gemessen=~3mm)| Anmerkung: mit dem Befehl "\captionabove" ändert sich die Reihenfolge der Tabellenüberschrift in: Fließtext-Tabellenüberschrift

% Intextsep! und Textfloat!
% Abstand Bild: Fließtext-Bild; Bildunterschrift-Fließtext
% Abstand Tabelle: Fließtext-Tabellenüberschrift; Tabelle-Fließtext
\setlength{\intextsep}{5.5mm plus0mm minus0mm} % Abstand für Gleitobjekte, die mit dem Parameter "[h]" platziert wurden (im PDF gemessen für Tabellen=~8mm; für Abbildungen=~6mm)
\setlength{\textfloatsep}{5.5mm plus0mm minus0mm} % Abstand für Gleitobjekte, die mit den Parametern "[t]" oder "[b]" platziert wurden
% \setlength{\textfloatsep}{7mm plus0mm minus0mm}

% Bildunterschrift für Subfloat!
\captionsetup[subfloat]{%
	labelformat=empty,%
	margin=0pt,% Einzug der Bildunterschrift von links
	% skip = 0pt,		% Abstand zwischen Bild und Bildunterschrift
	aboveskip=2mm,% Abstand zwischen Bild und Bildunterschrift
	belowskip=0mm,% Abstand zwischen Bildunterschrift und der nächsten Bildreihe sowie zwischen Bildunterschrift und Unterschrift der Abbildung / da es hier zu einer Überschneidung der Befehle "\captionsetup[subfloat]{belowskip}" und "\captionskip[figure]{skip}" kommt, muss der vertikale Abstand zwischen den Bildreihen mit dem Befehl "\vspace{3mm}" gesetzt werden
	% font={footnotesize,rm},%
	% labelfont={footnotesize,bf},%
	% format=hang,% Zweite und weitere Zeilen einrücken (an der ersten Zeile ausrichten)
	indention=0em,% Einruecken der Beschriftung
	labelsep=space,% "period, space, quad, newline"
	justification=RaggedRight,% Flattersatz mit Silbentrennung
	% justification=raggedright,% Flattersatz ohne Silbentrennung
	% justification=centering,% Zentriert
	% justification=justified,% Blocksatz
	singlelinecheck=true,% "false" (true = bei einer Zeile immer zentrieren)
	position=auto,% "top, bottom"
	labelformat=parens% "simple, empty" = wie die Bezeichnung gesetzt wird
}

%%%%%%%%%%%%%%%%%%%%%%%%%%%%%%%%%%%%%%%%%%%%%%%%%%
%%%  			Gleitobjekte!				   %%%
%%%			  "figure", "table"			   	   %%%
%%%				Layout, Größe  	   			   %%%
%%%%%%%%%%%%%%%%%%%%%%%%%%%%%%%%%%%%%%%%%%%%%%%%%%

% Mindestfüllgrad einer Seite mit einem Gleitobjekt
\renewcommand{\floatpagefraction}{0.7}

% Maximale Größe eines Gleitobjekts am unteren Seitenrand
\renewcommand{\topfraction}{0.8}

% Maximale Größe eines Gleitobjekts am oberen Seitenrand
\renewcommand{\bottomfraction}{0.8}

% Mögliche Abstandsvergrößerung innerhalb einer Zeile bei unschönem Zeilenumbruch
\setlength{\emergencystretch}{4pt}

% Mindestanteil an Text auf einer Seite mit Gleitobjekt
\renewcommand{\textfraction}{0.1}

%%%%%%%%%%%%%%%%%%%%%%%%%%%%%%%%%%%%%%%%%%%%%%%%%%
%%%  				Zeilenabstand!			   %%%
%%%%%%%%%%%%%%%%%%%%%%%%%%%%%%%%%%%%%%%%%%%%%%%%%%

% Zeilenabstände auf 1-fach festlegen
%\usepackage[singlespacing]{setspace}

% Zeilenabstände auf 1,15-fach festlegen
\usepackage{setspace}
\setstretch{1.15}

% Zeilenabstände auf 1,2-fach festlegen
%\usepackage{setspace}
%\setstretch{1.2}

% Zeilenabstände auf 1,5-fach festlegen
%\usepackage[onehalfspacing]{setspace}

%%%%%%%%%%%%%%%%%%%%%%%%%%%%%%%%%%%%%%%%%%%%%%%%%%
%%% 		Verzeichnisse!		     		   %%%
%%% 		Inhaltsverzeichnis!	     		   %%%
%%% 		Abbildungsverzeichnis!     		   %%%
%%% 		Tabellenverzeichnis!     		   %%%
%%%%%%%%%%%%%%%%%%%%%%%%%%%%%%%%%%%%%%%%%%%%%%%%%%

% toc = table of contents (Inhaltsverzeichnis)
% lof = list of figures (Abbildungsverzeichnis)
% lot = list of tables (Tabellenverzeichnis)

\makeatletter
%
% Automatischer Zeilenumbruch in Verzeichnissen (toc, lof, lot) bei langen Überschriften, horizontaler Abstand zum rechten Satzspiegelrand; "plus1fil": Keine Silbentrennung im Inhaltsverzeichnis, dadurch wird der Blocksatz für Verzeichnisse überwiegend außer Kraft gesetzt, weil keine Silbentrennung möglich
\renewcommand\@tocrmarg{7em plus1fil}%neu
% \renewcommand\@tocrmarg{4em plus1fil}%alt

% Abstand der Seitenzahl im Inhaltsverzeichnis zum letzten Punkt der gepunkteten Linie
\renewcommand\@pnumwidth{1em} % !Setzt bei 8pt oder 0em die Seitenzahl außerhalb des Satzspiegels (siehe "\usepackage{showframe}"!
%
% Ändert den Abstand zwischen den Punkten der gepunkteten Linie
% \renewcommand*{\@dotsep}{1.5}% Default ist 4.5
%
% \renewcommand*\l@chapter{\@dottedtocline{0}{1.5em}{2em}}
%
% \renewcommand*\l@figure{\@dottedtocline{1em}{0em}{2.3em}}% Standard ist {1.5em}{0em}{2.3em}
% \let\l@table\l@figure
%
\makeatother

% Einzug der Kapitelnummer (tocindent) und Abstand zwischen Kapitelnummer und Kapiteltext/Überschrift im Inhaltsverzeichnis (tocnumwidth)
\RedeclareSectionCommand[tocindent=0em,tocnumwidth=1.5em]{chapter}
\RedeclareSectionCommand[tocindent=1.5em,tocnumwidth=1.9em]{section}
\RedeclareSectionCommand[tocindent=3.4em,tocnumwidth=2.6em]{subsection}

% Abbildungsverzeichnis! Einzug und Abstand
\DeclareTOCStyleEntry[%
dynnumwidth=true,% Bei Bedarf wird der Abstand von "numsep" erweitert (z.B. bei zwei Ziffern) 
% entryformat=\sffamily\large,% Schriftart für Verzeichnis
indent=0pt,% Einzug links
% listname={Abbildungsverzeichnis},% Name des Abbildungsverzeichnisses
numsep=1em,% Abstand zwischen Nummer links und Text (Abbildungstitel), behält den Abstand zur Seitenzahl bei
numwidth=3em,% Abstand zwischen Text und Seitennummer rechts
% pagenumberbox=1em,% Abstand zwischen Nummer links und Text (Tabellentitel)
% pagenumberformat=\sfamily\large,% Schriftart für Seitennummer
% pagenumberwidth=1em,% Abstand zwischen Seitennummer und dem Text (links), schiebt auch die gepunktete Linie nach links
% entrynumberformat=\gobble,%
rightindent=8em,% Abstand zwischen Seitenzahl rechts und Text (Tabellentitel), Silbentrennung ist aktiv
]{default}{figure}

% Tabellenverzeichnis! Einzug und Abstand 
\DeclareTOCStyleEntry[%
% beforeskip=1.15em plus 1pt,%
dynnumwidth=true,% Bei Bedarf wird der Abstand von "numsep" erweitert (z.B. bei zwei Ziffern)
% entryformat=\sffamily\large,% Schriftart für Verzeichnis
indent=0pt,% Einzug der Nummerierung links
% linefill=\skillmon@chapter@dotfill,%
% listname={Tabellenverezichnis},% Name des Tabellenverzeichnisses
numsep=1em,% Abstand zwischen Nummer links und Text (Tabellentitel), behält den Abstand zu Seitenzahl bei
numwidth=3em,% Abstand zwischen Text und Seitennummer rechts
% pagenumberbox=1em,% Abstand zwischen Nummer links und Text (Tabellentitel)
% pagenumberwidth=1em,% Abstand zwischen Seitennummer und dem Text (links), schiebt auch die gepunktete Linie nach links
% entryformat=\textbf%
% pagenumberbox=\relax,%
% pagenumberformat=\ssfamily\large,% Schriftart für Seitennummer
% pagenumberformat=\usekomafont{tocentry},% Alternativ
% entrynumberformat=\gobble,%
rightindent=8em,% Abstand zwischen Seitenzahl rechts und Text (Tabellentitel), Silbentrennung ist aktiv
]{default}{table}

% \DeclareTOCStyleEntry[%
% dynnumwidth=true,% Bei Bedarf wird der Abstand von "numsep" erweitert (z.B. bei zwei Ziffern) 
% % entryformat=\sffamily\large,% Schriftart für Verzeichnis
% indent=0pt,% Einzug links
% % listname={Abbildungsverzeichnis},% Name des Abbildungsverzeichnisses
% numsep=1em,% Abstand zwischen Nummer links und Text (Abbildungstitel), behält den Abstand zur Seitenzahl bei
% numwidth=3em,% Abstand zwischen Text und Seitennummer rechts
% % pagenumberbox=1em,% Abstand zwischen Nummer links und Text (Tabellentitel)
% % pagenumberformat=\sfamily\large,% Schriftart für Seitennummer
% % pagenumberwidth=1em,% Abstand zwischen Seitennummer und dem Text (links), schiebt auch die gepunktete Linie nach links
% % entrynumberformat=\gobble,%
% ]{default}{listings}

% Schriftfamilie für Verzeichnisse (toc, lof, lot)
% \addtokomafont{disposition}{\sffamily} % Schrift ohne Serifen
% \addtokomafont{disposition}{\rmfamily} % Schrift mit Serifen

% Seitenzahlen im Inhaltsverzeichnis für Überschriften der Kategorie "\chapter"
\setkomafont{chapterentrypagenumber}{\rmfamily} % Seitenzahlen im Inhaltsverzeichnis mit Serifen
% \setkomafont{chapterentrypagenumber}{\rmfamily\mdseries} % Seitenzahlen im Inhaltsverzeichnis mit Serifen, aber nicht fett

% \addtokomafont{chapterentrypagenumber}{\normalfont\normalcolor\fontfamily{phv}\selectfont}

% Schriftfamilie für Überschriften der Kategorie "\chapter": "\rmfamily" ("\chapter" mit Serifen); "\sffamily" ("\chapter" ohne Serifen)
% \setkomafont{chapterentry}{\rmfamily\bfseries} 
% \setkomafont{chapterentry}{\sffamily} 

% Vereinheitlichung der Überschriften im Inhaltsverzeichnis. Ändert die Darstellung Überschriften der Kategorie "\chapter" im Inhaltsverzeichnis in die Kategorie "\section"; Anpassung der Schriftfamilie der Überschriften der Kategroie "\chapter": "\rmfamily" (Überschrift der Kategorie "\chapter" mit Serifen); "\sffamily" (Überschrift der Kategorie "\chapter" ohne Serifen)
% \setkomafont{sectioning}{\rmfamily\normalsize} 
% \setkomafont{sectioning}{\sffamily} 

\KOMAoptions{toc=chapterentrydotfill} % Auch bei Überschriften der Kategorie "\chapter" Punkte zur Seitenzahl im Inhaltsverzeichnis hinzufügen

% Bis Ebene 3 (subsection) im Inhaltsverzeichnis anzeigen
\setcounter{tocdepth}{2} 

% Eventuell ändern, um eine schönere Seitendarstellung zu erhalten
\BeforeStartingTOC{\setstretch{1.075}} 

% Schriftfamilie im Inhaltsverzeichnis auf Sans Serif stellen
% \newcommand*\tocentryformat[1]{{\sffamily#1}}
% \RedeclareSectionCommands
%   [
%     tocentryformat=\tocentryformat,
%     tocpagenumberformat=\tocentryformat
%   ]
%   {section,subsection,subsubsection,paragraph,subparagraph}

%%%%%%%%%%%%%%%%%%%%%%%%%%%%%%%%%%%%%%%%%%%%%%%%%%
%%% 		Inhaltsverzeichnis! (tocloft) 	   %%%
%%%%%%%%%%%%%%%%%%%%%%%%%%%%%%%%%%%%%%%%%%%%%%%%%%

% Inhaltsverzeichnis richtig darstellen
% \usepackage[titles]{tocloft}

% Auch bei den Kapiteln Punkte darstellen
% \renewcommand{\cftchapdotsep}{\cftdotsep}
% \renewcommand{\cftchapleader}{\cftdotfill{\cftchapdotsep}}

% Seitenzahlen bei Kapitel in serifenloser Schriftart darstellen
% \renewcommand{\cftchappagefont}{\fontfamily{phv}\normalsize\bfseries}

% Verzeichnisse aktualisieren
%Fonts im Inhaltsverzeichnis
% \renewcommand\cftchapfont{\fontfamily{phv}\normalsize\bfseries}
% \renewcommand\cftsecfont{\fontfamily{phv}\fontsize{11}{11}}

%Fonts in Kapiteln und sections...
%\renewcommand\cftchappagefont{\fontfamily{phv}\normalsize\bfseries}
% \renewcommand\cftsecpagefont{\fontfamily{phv}\fontsize{11}{11}}

%%%%%%%%%%%%%%%%%%%%%%%%%%%%%%%%%%%%%%%%%%%%%%%%%%
%%% 			Überschriften!   		  	   %%%
%%%%%%%%%%%%%%%%%%%%%%%%%%%%%%%%%%%%%%%%%%%%%%%%%%

% Auch die 4. Ebene nummerieren (subsubsection)
\setcounter{secnumdepth}{4} 

% Schriftarten und -größen für die Überschriften vorgeben
\addtokomafont{chapter}{\fontfamily{phv}\fontsize{20}{22}\bfseries} 		% z. B. "2 Stand der Technik" \fontsize{Schriftgröße 18 Pt}{Abstand vor der Überschrift: 20 Pt}
\addtokomafont{section}{\fontfamily{phv}\fontsize{15}{17}\bfseries}			% z. B. "2.1 Literatur und Forschung" \fontsize{Schriftgröße 14 Pt}{Abstand vor der Überschrift: 16 Pt}
\addtokomafont{subsection}{\fontfamily{phv}\fontsize{13}{15}\bfseries}		% z. B. "2.1.1 Disziplinäre Entwicklung" \fontsize{Schriftgröße 12 Pt}{Abstand vor der Überschrift: 14 Pt}
\addtokomafont{subsubsection}{\fontfamily{phv}\fontsize{10}{12}\bfseries}	% z. B. "2.1.1.1 Genese wissenschaftlicher" Konzepte" \fontsize{Schriftgröße 10 Pt}{Abstand vor der Überschrift: 12 Pt}

% Überschriften auslinieren
% Horizonaler Abstand zwischen Nummerierung und Überschrift
\renewcommand*{\chapterformat}{\makebox[1.4cm][l]{\thechapter\autodot}}
\renewcommand*{\sectionformat}{\makebox[1.4cm][l]{\thesection\autodot}}
\renewcommand*{\subsectionformat}{\makebox[1.4cm][l]{\thesubsection\autodot}}
\renewcommand*{\subsubsectionformat}{\makebox[1.4cm][l]{\thesubsubsection\autodot}}

%\RedeclareSectionCommand[beforeskip=-1.00\baselineskip,afterskip=0.50\baselineskip]{section}
%\RedeclareSectionCommand[beforeskip=-0.75\baselineskip,afterskip=0.50\baselineskip]{subsection}
%\RedeclareSectionCommand[beforeskip=-0.50\baselineskip,afterskip=0.25\baselineskip]{subsubsection}

% \DeclareFixedFont{\chapterfont}{T1}{phv}{bx}{n}{11cm}

%%%%%%%%%%%%%%%%%%%%%%%%%%%%%%%%%%%%%%%%%%%%%%%%%%
%%%  Bezeichnungen!: Abbildungen! / Tabellen!  %%%
%%%%%%%%%%%%%%%%%%%%%%%%%%%%%%%%%%%%%%%%%%%%%%%%%%

% Bezeichnung/Name für Abbildungen ändern
\addto\captionsngerman{\renewcommand{\figurename}{Abbildung}}
% \addto\captionsenglish{\renewcommand{\figurename}{Abbildung}}
% \newcaptionname{ngerman}\figurename{Abbildung}% 

% Bezeichnung/Name für Tabellen ändern
\addto\captionsngerman{\renewcommand{\tablename}{Tabelle}}
% \addto\captionsenglish{\renewcommand{\tablename}{Tabelle}}
% Tabellen Bezeichnung ändern
% \newcaptionname{ngerman}\tablename{Tabelle}

% Bezeichnung/Name für Listings ändern
\addto\captionsngerman{\renewcommand{\lstlistingname}{\latex-Quellcode}}

%%%%%%%%%%%%%%%%%%%%%%%%%%%%%%%%%%%%%%%%%%%%%%%%%%
%%% 				Schriftgröße!			   %%%
%%%				Kopfzeilen!, Fußzeilen!, 	   %%%
%%%				Seitenzahl!, Textfarbe!,   	   %%%
%%%%%%%%%%%%%%%%%%%%%%%%%%%%%%%%%%%%%%%%%%%%%%%%%%

% Größen der Beschriftungen vorgeben
%\addtokomafont{caption}{\footnotesize}
%\setkomafont{captionlabel}{\footnotesize}

% Größe der Kopf- und Fußzeile vorgeben
\setkomafont{pageheadfoot}{\footnotesize} 

% Größe der Seitenzahl
\setkomafont{pagenumber}{\normalsize}

% Farben im Dokument zulassen
\usepackage{color}
\usepackage{xcolor} % Für die Farbe "gray" notwendig

% Textfarbe schwarz definieren
\color[cmyk]{0,0,0,1}

%%%%%%%%%%%%%%%%%%%%%%%%%%%%%%%%%%%%%%%%%%%%%%%%%%
%%% 			Schriftarten!			  	   %%%
%%%%%%%%%%%%%%%%%%%%%%%%%%%%%%%%%%%%%%%%%%%%%%%%%%

%%%%%%%%%%%%%%%%%%%%%%%%%%%%%%%%%%%%%%%%%%%%%%%%%%
%%% 			Mit Serifen			  	   	   %%%
%%%%%%%%%%%%%%%%%%%%%%%%%%%%%%%%%%%%%%%%%%%%%%%%%%

% Nimbus 15 Serif
% Beispiele zum Schriftbild (Typografie)siehe: https://tug.org/FontCatalogue/nimbus15serif/
\usepackage{nimbusserif}

% URW Nimbus Roman (ähnlich Times New Roman)
% Beispiele zum Schriftbild (Typografie)siehe: https://tug.org/FontCatalogue/urwnimbusroman/ 
%\usepackage{mathptmx}

% Utopia Regular with Fourier
% Beispiele zum Schriftbild (Typografie)siehe: https://tug.org/FontCatalogue/utopia-fouriermath/ 
%\usepackage{fourier}

% Utopia Regular with Math Design
% Beispiele zum Schriftbild (Typografie)siehe: https://tug.org/FontCatalogue/utopia-mathdesign/ \usepackage[adobe-utopia]{mathdesign}

%%%%%%%%%%%%%%%%%%%%%%%%%%%%%%%%%%%%%%%%%%%%%%%%%%
%%% 			Ohne Serifen		  	   	   %%%
%%%%%%%%%%%%%%%%%%%%%%%%%%%%%%%%%%%%%%%%%%%%%%%%%%

% Helvetica-Klon (phv) als Standardschrift für serifenlose Texte (Überschriften und Überschriften im Inhaltsverzeichnis) verwenden
\renewcommand{\sfdefault}{phv}

% URW Nimbus Sans
% Beispiele zum Schriftbild (Typografie)siehe: https://tug.org/FontCatalogue/urwnimbussans/ 
%\usepackage[scaled]{helvet}
%\renewcommand*\familydefault{\sfdefault}

% Nimbus 15 Sans
% Beispiele zum Schriftbild (Typografie)siehe: https://tug.org/FontCatalogue/nimbus15sans/ 
%\usepackage{nimbussans}
%\renewcommand*\familydefault{\sfdefault}

%%%%%%%%%%%%%%%%%%%%%%%%%%%%%%%%%%%%%%%%%%%%%%%%%%
%%%        			Microtype!			   	   %%%
%%%%%%%%%%%%%%%%%%%%%%%%%%%%%%%%%%%%%%%%%%%%%%%%%%
% \usepackage[stretch=10,shrink=10]{microtype} % Verhindert Unschärfe und Verschwimmen der Schrift und reduziert die Anzahl der Badboxes (underfull/overfull); muss nach der Schriftart eingebunden werden

%%%%%%%%%%%%%%%%%%%%%%%%%%%%%%%%%%%%%%%%%%%%%%%%%%
%%%        			Kopfzeilen!			   	   %%%
%%%        			Kolumnentitel!		   	   %%%
%%%%%%%%%%%%%%%%%%%%%%%%%%%%%%%%%%%%%%%%%%%%%%%%%%

% Paket für das automatische Setzen von Kopf- und Fußzeilen
\usepackage[%
markcase=ignoreuppercase,%
automark,%
autooneside=false%
]{scrlayer-scrpage}
%\usepackage[nouppercase]{scrpage2} % Obsoletes Paket, das durch das Paket "\usepackage{scrlayer-scrpage}" ersetzt wurde

% Linienstärke in der Kopfzeile definieren
\KOMAoptions{headsepline=0.5pt}
%\setheadsepline{0.5pt} % Befehlszeile für das obsolete Paket "\usepackage{scrpage2}"

% Abstand zwischen Textkörper und Linie in der Kopfzeile
\setlength{\headsep}{8mm}

%%%%%%%%%%%%%%%%%%%%%%%%%%%%%%%%%%%%%%%%%%%%%%%%%%
%%% 				Notizen!	  	   	   	   %%%
%%%%%%%%%%%%%%%%%%%%%%%%%%%%%%%%%%%%%%%%%%%%%%%%%%

% Erlaubt das Einfügen von Notizen
%\usepackage{todonotes}

% Deaktiviert alle eingefügten Notizen
%\usepackage[disable]{todonotes}

% Einfügen von Notizen im Fließtext
%\todo[inline]{Das ist eine Notiz}

%%%%%%%%%%%%%%%%%%%%%%%%%%%%%%%%%%%%%%%%%%%%%%%%%%
%%% 			Links! (url, hyperref)		   %%%
%%%%%%%%%%%%%%%%%%%%%%%%%%%%%%%%%%%%%%%%%%%%%%%%%%

% Darstellung URL mit Zeilenumbruch
\usepackage[hyphens]{url}

% Zeilenumbrüche in URLs nach folgenden Zeichen
\appto\UrlBreaks{\do\a\do\b\do\c\do\d\do\e\do\f\do\g\do\h\do\i\do\j\do\k\do\l\do\m\do\n\do\o\do\p\do\q\do\r\do\s\do\t\do\u\do\v\do\w\do\x\do\y\do\z\do\/\do\.}

% Darstellung und Verlinkungen im pdf-Dokument einstellen
\usepackage[%
hidelinks,% Links als normaler Text darstellen
% colorlinks, % Links in Farbe darstellen
% citecolor=blue, % Quellenangaben im Fließtext in der ausgewählten Schriftfarbe darstellen
pdfpagemode=UseNone,% Lesezeichen im PDF-Reader nicht anzeigen
pdfpagelayout=TwoColumnRight,% Seitenanzeige des PDF-Dokuments angeben
pdfauthor={\autor},% Autor des PDF-Dokuments
pdftitle={\pdftitle},% Titel des PDF-Dokuments
bookmarksnumbered=true,% Nummerierte Kapitel auch in der PDF-Navigation nummerieren
]{hyperref}
	
%%%%%%%%%%%%%%%%%%%%%%%%%%%%%%%%%%%%%%%%%%%%%%%%%%
%%% 	Einstellungen weiterer Pakete		   %%%
%%%%%%%%%%%%%%%%%%%%%%%%%%%%%%%%%%%%%%%%%%%%%%%%%%

% Ansicht Literaturverzeichnis
%\bibliographystyle{plaindin}

% Einbinden von Bildern ermöglichen
\usepackage{graphicx}	

% Gedrehte Objekte ermöglichen
\usepackage{rotating}

% Erweiterte Tabellenumgebung
\usepackage{tabularx}

% Erweiterte Flattersatz-Kommandos
\usepackage{ragged2e}

% Linksbündiger Flattersatz in den Bezeichnungen
%\usepackage[justification=RaggedRight]{caption}
%\usepackage[justification=justified]{caption}
%\captionsetup[subfigure]{justification=RaggedRight}

% Neuer Spaltentyp "L" mit Breitenangabe für linksbündigen Flattersatz
\newcolumntype{L}[1]{>{\RaggedRight\arraybackslash}p{#1}}

%%%%%%%%%%%%%%%%%%%%%%%%%%%%%%%%%%%%%%%%%%%%%%%%%%
%%% 			Mathematik!		 			   %%%
%%%%%%%%%%%%%%%%%%%%%%%%%%%%%%%%%%%%%%%%%%%%%%%%%%

% Mathematische Symbole
\usepackage{amsmath}
\usepackage{amssymb}
\usepackage{amsfonts}

% Zeilen in Tabellen können verbunden werden
\usepackage{multirow}

% Zusätzliche Textsymbole zur Verfügung stellen
\usepackage{textcomp}

% Operatorensymbole definieren
\newcommand{\real}{\operatorname{Re}}				% Realteil
\newcommand{\opdiv}{\operatorname{div}}				% Divergenzoperator
\newcommand{\rot}{\operatorname{rot}}				% Rotationsoperator
\newcommand{\grad}{\operatorname{grad}}				% Gradientenoperator
\newcommand{\imag}{\operatorname{Im}}				% Imaginärteil
\newcommand{\imein}{\operatorname{j}}				% Imaginäre Einheit "j"

% Erweiterte Listenanweisungen
\usepackage{etoolbox}

% Einzug im Abbildungsverzeichnis zu null setzen (erfordert das Paket "tocloft")
% \renewcommand{\cftfigindent}{0cm}

% Einzug im Tabellenverzeichnis zu null setzen (erfordert das Paket "tocloft")
% \renewcommand{\cfttabindent}{0cm}
	
% Abkürzungsverzeichnis erstellen in deutscher bzw. englischer Sprache
%\usepackage[english]{nomencl}
\usepackage[german]{nomencl}

% Befehl für einen Eintrag im Abkürzungsverzeichnis in "\sym" umbennen
\let\sym\nomenclature

% Name des Abkürzungsverzeichnis ändern
\renewcommand{\nomname}{Abkürzungs- und Symbolverzeichnis}

% Spaltenbreite der Formelzeichen auf "20 %" der Textbreite setzen
\setlength{\nomlabelwidth}{.2\textwidth}

% Einheiten in die Bezeichnung mit aufnehmen und rechtsbündig setzen
\newcommand{\nomunit}[1]{\renewcommand{\nomentryend}{\hspace*{\fill}#1}}

% Zeilenabstände verkleineren auf normalen Textabstand
\setlength{\nomitemsep}{-\parsep}

% Abkürzungsverzeichnis erzeugen
\makenomenclature

% Weitere Verzeichnisse erzeugen
\makeindex

% Name des Formelverzeichnisses ändern
\AtBeginDocument{% 
  \newcaptionname{ngerman}\equationname{Formel}% 
  \newcaptionname{ngerman}\listequationname{Formelverzeichnis}% 
  \addtocontents{toc}{\protect\activateonlyattoc}% Z. B. für Umbrüche von langen Überschriften im Inhaltsverzeichnis mit dem Befehl \onlyattoc{\protect\\} (Beispiel: \chapter{Lange Kapitelüberschriften und der manuelle Zeilenumbruch \onlyattoc{\protect\\} für eine saubere Darstellung im Text})
}

\DeclareRobustCommand*{\onlyattoc}[1]{} % Z. B. für Umbrüche von langen Überschriften im Inhaltsverzeichnis mit dem Befehl \onlyattoc{\protect\\} (Beispiel: \chapter{Lange Kapitelüberschriften und der manuelle Zeilenumbruch \onlyattoc{\protect\\} für eine saubere Darstellung im Text})
\newcommand*{\activateonlyattoc}{\DeclareRobustCommand*{\onlyattoc}[1]{##1}}% Z. B. für Umbrüche von langen Überschriften im Inhaltsverzeichnis mit dem Befehl \onlyattoc{\protect\\} (Beispiel: \chapter{Lange Kapitelüberschriften und der manuelle Zeilenumbruch \onlyattoc{\protect\\} für eine saubere Darstellung im Text})

% Formatierung für Formelverzeichnis!
\DeclareNewTOC[% 
  tocentryindent=0pt,%
  tocentrynumwidth=2em,%
  hang=1.5em,% 
  type=equation,%
  name={Gl.},% 
  types=equations,% 
  listname={Formelverzeichnis},% 
]{equ} 
\newcommand{\equationentry}[2][\theequation]{% 
  \addxcontentsline{equ}{equation}[{#1}]{\kern 1em #2}% 
} 
\BeforeStartingTOC[equ]{\def\autodot{:}}

%%%%%%%%%%%%%%%%%%%%%%%%%%%%%%%%%%%%%%%%%%%%%%%%%%
%%%  				Index!					   %%%
%%%%%%%%%%%%%%%%%%%%%%%%%%%%%%%%%%%%%%%%%%%%%%%%%%

% Ermöglicht die Nutzung eines Index bzw. Stichwortverzeichnisses
\usepackage{makeidx}
% Beispiel: Das ist ein Eintrag\index{Eintrag} in den Index.
% Setze "\printindex" an die entsprechende Stelle in der Datei "KSP_Diss_A5.tex"

%%%%%%%%%%%%%%%%%%%%%%%%%%%%%%%%%%%%%%%%%%%%%%%%%%
%%%  				Listings!				   %%%
%%%%%%%%%%%%%%%%%%%%%%%%%%%%%%%%%%%%%%%%%%%%%%%%%%

% Ermöglicht das Ausgeben von LaTeX-Quellcode im Text
\usepackage{listings}
% \usepackage{listingsutf8}

% Umlaute und Sonderzeichen in Listings ausgeben
\lstset{literate=%
{ä}{{\"a}}1 
{ë}{{\"e}}1 
{ï}{{\"i}}1 
{ö}{{\"o}}1 
{ü}{{\"u}}1
{Ä}{{\"A}}1 
{Ë}{{\"E}}1 
{Ï}{{\"I}}1 
{Ö}{{\"O}}1 
{Ü}{{\"U}}1
{á}{{\'a}}1 
{é}{{\'e}}1 
{í}{{\'i}}1 
{ó}{{\'o}}1 
{ú}{{\'u}}1
{Á}{{\'A}}1 
{É}{{\'E}}1 
{Í}{{\'I}}1 
{Ó}{{\'O}}1 
{Ú}{{\'U}}1
{à}{{\`a}}1 
{è}{{\`e}}1 
{ì}{{\`i}}1 
{ò}{{\`o}}1 
{ù}{{\`u}}1
{À}{{\`A}}1 
{È}{{\'E}}1 
{Ì}{{\`I}}1 
{Ò}{{\`O}}1 
{Ù}{{\`U}}1
{â}{{\^a}}1 
{ê}{{\^e}}1 
{î}{{\^i}}1 
{ô}{{\^o}}1 
{û}{{\^u}}1
{Â}{{\^A}}1 
{Ê}{{\^E}}1 
{Î}{{\^I}}1 
{Ô}{{\^O}}1 
{Û}{{\^U}}1
{œ}{{\oe}}1 
{Œ}{{\OE}}1 
{æ}{{\ae}}1 
{Æ}{{\AE}}1 
{ß}{{\ss}}1
{ű}{{\H{u}}}1 
{Ű}{{\H{U}}}1 
{ő}{{\H{o}}}1 
{Ő}{{\H{O}}}1
{ç}{{\c c}}1 
{Ç}{{\c C}}1
{ã}{{\~a}}1 
{å}{{\r a}}1 
{Å}{{\r A}}1
{ø}{{\o}}1 
{€}{{\EUR}}1 
{£}{{\pounds}}1
{~}{{\textasciitilde}}1
}

% Abstand für Listings und Captions innerhalb einer Listing-Umgebung
\lstset{%
aboveskip=5mm,
belowskip=1mm,
abovecaptionskip=0mm,
belowcaptionskip=3mm,
% escapechar=|
}

% Stil für Listings ohne Zeilennummern
\lstdefinestyle{kspnonumbers}{%
language=[LaTeX]TeX,
escapechar=|,
commentstyle=\color{gray},
keywordstyle=\color{magenta},
stringstyle=\color{blue},
% backgroundcolor=\color{gray},
% inputpath=Ordnername
basicstyle=\ttfamily\small\bfseries,
breaklines=true,
% breakatwhitespace=true,
breakindent=0pt,
showstringspaces=false,
% captionpos=b, % Beschriftung unterhalb des Listings
tabsize=2,
frame=tbrl,% Top ("T","t"), buttom ("B", "b"), left ("L", "l"), right ("R", "r")
% frame=single,
framesep=0em,% Breite des Rahmens über den Satzspiegelrand
framextopmargin=5pt,
framexleftmargin=5pt,
framexrightmargin=5pt,
framexbottommargin=5pt,
% xleftmargin=2em,% Abstand vom linken Satzspiegelrand nach innen
% xrightmargin=2em,% Abstand vom rechten Satzspeigelrand nach innen
% numbers=left,
% numberstyle=\footnotesize\color{gray},
% numbersep=10pt,
% stepnumber=2,
morekeywords={%
chapter
}
}

% % Stil für Listings mit Zeilennummern
% \lstdefinestyle{kspnumbers}{%
% % inputpath=Ordnername
% basicstyle=\ttfamily\small,
% % backgroundcolor=\color{gray},
% breaklines=true,
% commentstyle=\color{gray},
% keywordstyle=\color{magenta},
% stringstyle=\color{blue}
% % captionpos=b,
% showstringspaces=false,
% numbers=left,
% % numberstyle=\footnotesize\color{gray}
% numbersep=10pt,
% stepnumber=2,
% tabsize=2,
% frame=tblr,% Top, buttom, left, right
% }

%%%%%%%%%%%%%%%%%%%%%%%%%%%%%%%%%%%%%%%%%%%%%%%%%%
%%%  				Latex!					   %%%
%%%%%%%%%%%%%%%%%%%%%%%%%%%%%%%%%%%%%%%%%%%%%%%%%%

% Neuer Befehl für die Ausgabe des LaTeX-Logos
\usepackage{xspace}
\newcommand{\latex}{\LaTeX\xspace}
\newcommand{\tex}{\TeX\xspace}



% Literaturverzeichnis
%-----------------------
%
% Kohm 2020: Kohm, Markus; Neukamp, Frank; Kielhorn, Axel (2020): Die Anleitung. KOMA-Script. Markus Kohm. 2020-03.12. (Online)