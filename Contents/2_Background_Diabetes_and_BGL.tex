\markboth{Background: Diabetes and Blood Glucose Monitoring}{Background: Diabetes and Blood Glucose Monitoring}

\chapter{Background: Diabetes and Blood Glucose Monitoring}
\label{cha:Background Diabetes and Blood Glucose Monitoring}

\section{Medical Context}
\label{sec:Medical context}
Diabetes mellitus (I will be referring to it as diabetes but diebetes mellitus is the medical term) is a metabolic disease, involving inappropriately elevated blood 
glucose levels (hyperglycemia).\cite{sapra_diabetes_2023} It can lead to severe complications, such as cardiovascular disease, kidney damage, nerve damage, 
eye and oral complications.\cite{noauthor_diabetes_2025-1} Diabetes can also develop when the body 
of a person isn’t responding to the effects of insulin properly. Diabetes affects people of all ages and most forms of diabetes are chronic.

The most common types of diabetes are Type 2, Prediabetes, Type 1 and Gestational Diabetes. The most common of these is Type 2 Diabetes. This is the type where the body
doesn't respond to insulin poperly or the body doesn't produce enough insulin. It is possible though that both is true for a person. Prediabetes is a condition where 
blood glucose levels are higher than usual, but not as high as to be diagnosed with Type 2 diabetes.\cite{noauthor_information_2023} Type 1 diabetes on the other hand 
is an autoimmune disease, which is a malfunction of the body's immune system that causes the body to attack its own 
tissues.\cite{noauthor_information_2023, fernandez_autoimmune_2024} In this case the immune system attacks insulin producing cells in the pancreas with up to 10\% of 
people having diabetes, having Type 1 diabetes. Another form being Gestational diabetes that develops and usually goes away during pregnancy. But people that had 
Gestational diabetes are at a greater risk of developing Type 2 diabetes later in life.\cite{noauthor_information_2023}

\section{Traditional invasive measurement techniques}
\label{sec:Traditional invasive measurement techniques}
When left unmanaged and untreated, diabetes causes serious health problems, as discribed earlier. So it is crucial to manage diabetes where monitoring blood glucose 
levels is essential. Current/traditional techniques to measure blood glucose levels are invasive. The most common method to measure the blood glucose level is with a
glucose meter, or glucometer. This is a small and portable machine that can measure a person's blood glucose level, requiring only a small sample of blood. There are 
multiple ways to collect the blood sample but the most common one is to prick the finger with a small needle. Other test sites are the upper arm, forearm, base of the 
thumb or the thigh. But readings in the fingertip are much more accurate so preferred. \cite{ellis_blood_2024} 

Another method is continuous glucose monitoring where how
the name already suggests the glucose levels are monitored constantly. The sensor for the monitoring is either inserted under the skin (a small needle) and held in place
with a stick patch (disposable sensor) or it is placed fully under the skin (implantable sensor). These sensors then transmit the data to a reciever, which more often
is a mobile phone. There a person can see its glucose levels, trends and get alarms, if the blood glucose level is too low or high.\cite{wong_continuous_2023}

\section{Need for non-invasive approaches}
\label{sec:Need for non-invasive approaches}
While continuous glucose monitoring (CGM) offers significant advantages over periodic finger-prick testing, such as enabling easier management of blood glucose levels and 
reducing the incidence of acute glycemic emergencies, the invasive nature of traditional methods remains a barrier to widespread and sustained use. Disposable CGM 
sensors must typically be replaced every 7 to 14 days, and implantable variants can last up to 180 days.\cite{wong_continuous_2023}

However, these conventional and minimally invasive methods are often painful, can be costly, and may discourage consistent monitoring, leading to poor adherence to 
testing routines \cite{maldocsda_non-invasive_2024}. By contrast, non-invasive glucose monitoring approaches hold promise for daily and continuous use by being 
painless, more comfortable, and potentially less costly.

Such innovations could significantly improve patient compliance and quality of life, addressing the limitations of traditional monitoring modalities.
\cite{owida_non-invasive_2022, ghosh_evolution_2025} Ultimately, non-invasive technologies may deliver effective, user-friendly alternatives that facilitate 
better long-term management of diabetes.