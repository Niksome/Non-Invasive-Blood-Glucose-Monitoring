% Version 1.2.0 EN
%
%%%%%%%%%%%%%%%%%%%%%%%%%%%%%%%%%%%%%%%%%%%%%%%%%%
%%%        			Preamble!            				   %%%
%%%%%%%%%%%%%%%%%%%%%%%%%%%%%%%%%%%%%%%%%%%%%%%%%%

%%%%%%%%%%%%%%%%%%%%%%%%%%%%%%%%%%%%%%%%%%%%%%%%%%
%%%        		Preamble / compile!       		   %%%
%%%%%%%%%%%%%%%%%%%%%%%%%%%%%%%%%%%%%%%%%%%%%%%%%%

% Important:
% To avoid errors and problems when compiling, please always use the current version of your operating system (Windows / Mac / Linux), the current version of your LaTeX distribution (MiKTeX / MacTeX / Tex Live) and the current version of your typesetting program or LaTeX editor (TeXstudio / Texmaker / TeXworks / TeXnicCenter). Update your operating system, your LaTeX distribution and your LaTeX editor before compiling this template or have them updated by your department IT representative.
%
% Notes on compile / setting order:
% For the correct display of the layout and the bibliographies (multibib) please compile as follows (see "Order of compilation operations"):
%
%- Windows 10: via the PowerShell or command prompt (cmd).
% To do this, go to the folder of this project, do not select any file, click on an empty space with the left mouse button and open the context menu while holding down the Shift key (Shift key + right mouse click) and select "Open PowerShell window here". Then enter "cmd" and call the compile processes individually as follows (see "Order of compilation operations"):
%
%- macOS Big Sur 11.0.1: via the terminal.
% To do this, go to the parent folder of this project and select the folder in which the project is located. Now open the context menu (secondary click) by pressing Control + mouse button or tapping with two fingers on the trackpad. Now select "New Terminal at Folder" and call up the compile processes individually as follows:
%
%
%- Order of compilation operations
% pdflatex KSP_Diss_17x24
% bibtex KSP_Diss_17x24
% bibtex journal				    % The file name of the file to be executed with Bibtex is specified in the command "\ newcites {}" in this TeX file
% bibtex conference				% The file name of the file to be executed with Bibtex is specified in the command "\ newcites {}" in this TeX file
% pdflatex KSP_Diss_17x24
% pdflatex KSP_Diss_17x24
%
%
% Done. 

%%%%%%%%%%%%%%%%%%%%%%%%%%%%%%%%%%%%%%%%%%%%%%%%%%
%%%        			Peamble / Documentclass!       %%%
%%%%%%%%%%%%%%%%%%%%%%%%%%%%%%%%%%%%%%%%%%%%%%%%%%

% Loads the document class "scrbook" (KOMA-Script-Book) with the properties: no dots after the numbering, font size 10pt, language packs: English
\documentclass[%
fontsize=10pt,% Font size 10 points
numbers=noenddot,% No end point after the numbering (i.e. 3.1 instead of 3.1.)
% chapteratlists=0pt,% toc, lot, lof: Set the vertical distance between the entries to 0
toc=listof,% Sets the table of figures (lof) and the table of tables (lot) in the table of contents (toc)
% toc=bibliography,% Places the bibliography in the table of contents (does not work with multiple bibliographies, even if the commands in the file "Contents/Bibliography.tex" are commented out: "\addcontentsline{toc}{section}{Journalartikel}", "\addcontentsline{toc}{section}{Konferenzbeiträge}", "\addcontentsline{toc}{chapter}{Literaturverzeichnis")
toc=chapterentrywithdots,% Adds a dotted line to the page number in the table of contents for headings in the "\chapter" category
% headings=optiontotocandhead,% To use the command "\addchap[tocentry={unnumbered in the table of contents},head={unnumbered in the table of contents}]{unnumbered}{}"
% headinclude,% 
% twoside,% Default for this document class "scrbook"
listof=nochaptergap,% No vertical spacing between the entries (chapter by chapter) in the list of figures or tables
% toc=flat% Reduces the space between the number on the left and text to a minimum; Subchapters are not indented
% toc=chapterentrydots=true,% KOMA guide, p. 74
% toc=sectionentrydots=true% KOMA guide, p. 74
]{scrbook} % Loads the document class "scrbook" (KOMA-Script-Book by Markus Kohm et al.)

%\documentclass[halfparskip,numbers=noenddot,a5paper,10pt,english,ngerman]{scrbook}


% Input, including umlauts (ä, ö, ü and ß) using the parameter [utf8]
\usepackage[utf8]{inputenc}

% Output, including umlauts
\usepackage[T1]{fontenc}

% Load the language packs for hyphenation
\usepackage[english]{babel}

% Insert more packages here
\usepackage{booktabs}
\usepackage{wrapfig}
\usepackage[all]{nowidow} % Prevents widows
\usepackage{amsmath}
\usepackage{amsfonts}
\usepackage{amssymb}
\usepackage{graphicx}
\usepackage[]{acronym}
\usepackage{trfsigns}
\usepackage{caption} % For subfigures
\usepackage{subcaption} % For "subfloat" environments (multi-image graphics / multiple images)
%\usepackage{subfigure} % Old package for "subfigures"
\usepackage{enumitem}
\usepackage{calc}
\usepackage{multirow}
\usepackage{textcmds} % For quotation marks
\usepackage{blindtext} % Enable blind text to display text in the template
\usepackage{fancyvrb} % For Verbatim environments to output LaTeX source code
%\usepackage{showframe} % Show margins
\usepackage{todonotes}
\usepackage{svg}

%%%%%%%%%%%%%%%%%%%%%%%%%%%%%%%%%%%%%%%%%%%%%%%%%%
%%%    		Preamble / Hyphenation!    	     	   %%%
%%%%%%%%%%%%%%%%%%%%%%%%%%%%%%%%%%%%%%%%%%%%%%%%%%

\hyphenation{
Ge-schich-te
An-ten-ne
% Please add words that should not be separated into this list without a hyphen
}

%%%%%%%%%%%%%%%%%%%%%%%%%%%%%%%%%%%%%%%%%%%
%%%    Preamble / Cite!  / Quote!       %%%
%%%%%%%%%%%%%%%%%%%%%%%%%%%%%%%%%%%%%%%%%%%

% Both \citep and \citet are defined by natbib and are thus not standard. The standard LATEX command \cite should be avoided, because it behaves like \citet for author–year citations, but like \citep for numerical ones
% Quoting from the bibliography "Literature.bib": "\citet{<ID>}" respectively "\citep{<ID>}"
% Quoting from the bibliography "Own_journal_papers.bib": "\citetjournal{<ID>}" respectively "\citepjournal{<ID>}"
% Quoting from the bibliography "Own_conference_papers.bib": "\citetconference{<ID>}" respectively "\citepconference{<ID>}"

% Quotation marks
% You form the English introductory quotation marks in LaTeX with two acute accents one after the other, i.e. with the characters `` (shift+´).
% Complete your English quote with two apostrophes in a row. So you write the sign '' (shift+#).

% By default, all titles / references are output in the bibliographies. If only the quoted titles / references are to be output in the bibliography, the command "\nocite{*}" or "\nocitejournal{*} or "\nociteconference{*} "must be commented out in the file "./Inhalt/Literaturverzeichnis.tex"

%\usepackage{cite}
\usepackage[%
round,% (Default) for round parentheses;
%square,% For square brackets
%curly,% For curly braces
%semicolon, % (Default), trennt mehrere Autoren mit Semicolon
%comma, % (Default) to separate multiple citations with semi-colons
%authoryear, % (Default) for author–year citations
%numbers, % For numerical citations
%super, % For superscripted numerical citations, as in Nature
]{natbib}
%
\bibpunct{(}% Left paranthesis
{)}% Right paranthesis
{,}% Separation between several titles in the text
{a}% Citation style: n=for numerical style, s=for numerical superscript style, a=any other letter for author–year, (default) author–year
{}% Punctuation before the year
{,}% Punctuation between different publication years of titles by the same author
%
% Cite!
% The following examples are taken from Patrick W. Daly (2010) "Natural Sciences Citations and References. Author-Year and Numerical Schemes"
%\citet{jon90} --> Jones et al. (1990)
%\citet[chap.~2]{jon90} --> Jones et al. (1990, chap. 2)
%\citep{jon90} --> (Jones et al., 1990)
%\citep[chap.~2]{jon90} --> (Jones et al., 1990, chap. 2)
%\citep[see][]{jon90} --> (see Jones et al., 1990)
%\citep[see][chap.~2]{jon90} --> (see Jones et al., 1990, chap. 2)
%\citet*{jon90} --> Jones, Baker, and Williams (1990)
%\citep*{jon90} --> (Jones, Baker, and Williams, 1990)
%
%\citet{jon90,jam91} --> Jones et al. (1990); James et al. (1991)
%\citep{jon90,jam91} --> (Jones et al., 1990; James et al. 1991)
%\citep{jon90,jon91} --> (Jones et al., 1990, 1991)
%\citep{jon90a,jon90b} --> (Jones et al., 1990a,b)

%%%%%%%%%%%%%%%%%%%%%%%%%%%%%%%%%%%%%%%%%%%%%%%%%%
%%%       Preamble /  Bibliography!            %%%
%%%%%%%%%%%%%%%%%%%%%%%%%%%%%%%%%%%%%%%%%%%%%%%%%%

\usepackage[resetlabels]{multibib} % Each bibliography begins with counter [1], provided that "\ bibliographystyle {plain}" is selected as the bibliography style
\usepackage{multibib} % The bibliographies counter is not reset (counting starting with [1] in "Journal articles" and will be continued in "Conference contributions"), provided that "\ bibliographiestyle {plain}" is selected as the bibliography style, with a different bibliography style, e.g. "\ bibliographystyle {alpha} no counter is used
\newcites{journal}{Bibliography/Own_journal_papers}
\newcites{conference}{Bibliography/Own_conference_papers}
%
\usepackage{etoolbox}
\BeforeBeginEnvironment{thebibliography}{% Redefine to insert your own publications without breaking
  \let\origchapter\chapter % Save the original definition of "\chapter"
  \let\chapter\section % Make \chapter behave like "\section"
}
\AfterEndEnvironment{thebibliography}{%
  \let\chapter\origchapter % Restore the original definition of "\chapter"
}

%%%%%%%%%%%%%%%%%%%%%%%%%%%%%%%%%%%%%%%%%%%%%%%%%%
%%%        Preamble / Title!          		     %%%
%%%%%%%%%%%%%%%%%%%%%%%%%%%%%%%%%%%%%%%%%%%%%%%%%%
\usepackage[scaled]{helvet}
\renewcommand{\familydefault}{\sfdefault}

\usepackage{graphicx,xcolor}
\usepackage{tikz}
\usetikzlibrary{calc}
% PDF-Title
\newcommand{\pdftitle}{Non-Invasive\ Blood\ Glucose\ Monitoring\ in\ Ears}

% Author
\newcommand{\autor}{Andrej\ Vladimirovič\ Ermoshkin}

% Most of the preamble is moved to the "dokOptions.tex" file; The file must be integrated at this point
% Settings such as scaling, font, spacing and the packages "geometry", "figure", "captions" are managed in the following file
% Version 1.2.0 EN
%
% Show page limits to check the page display
%\usepackage{showframe}

%%%%%%%%%%%%%%%%%%%%%%%%%%%%%%%%%%%%%%%%%%%%%%%%%%
%%% 		Manual line break in the 		   %%%
%%%			list of figures and tables		   %%%
%%%%%%%%%%%%%%%%%%%%%%%%%%%%%%%%%%%%%%%%%%%%%%%%%%

% Set line breaks manually in \ listoffigures and \ listoftables with "\ protect \\" (without quotation marks) in the optional parameter of the "\ caption" command
% Example: 
%\caption[Exterior view of the KIT library. Consetetur sadipscing elitr,\protect\\ sed diam nonumy eirmod tempor invidunt ut labore]{Exterior view of the KIT library. Consetetur sadipscing elitr, sed diam nonumy eirmod tempor invidunt ut labore}

%%%%%%%%%%%%%%%%%%%%%%%%%%%%%%%%%%%%%%%%%%%%%%%%%%
%%% 			Special character			  %%%%
%%%%%%%%%%%%%%%%%%%%%%%%%%%%%%%%%%%%%%%%%%%%%%%%%%

% Non-breaking space, protected space: ~
% Non-breaking narrow space, protected narrow space: \.
% Backslash: \textbackslash
% Percent sign: \%
% Ampersand: \&
% Circumflex: \^
% Tilde: \~
% Dollar: \$
% Hashtag, number sign: \#
% Low line: \_
% Curly bracket: \{
% Non-breaking hyphen: "~ % Requires the package "\usepackage[english]{babel}"
% Non-breaking hyphen alternative: \hbox{-}
% Soft hyphen: \-

%%%%%%%%%%%%%%%%%%%%%%%%%%%%%%%%%%%%%%%%%%%%%%%%%%
%%% 			General Settings			  %%%%
%%%%%%%%%%%%%%%%%%%%%%%%%%%%%%%%%%%%%%%%%%%%%%%%%%

% Adjust margins
\setlength{\topskip}{10.5pt} % Prevent an error message

%%%%%%%%%%%%%%%%%%%%%%%%%%%%%%%%%%%%%%%%%%%%%%%%%%
%%%        			Scaling					  %%%%
%%%%%%%%%%%%%%%%%%%%%%%%%%%%%%%%%%%%%%%%%%%%%%%%%%

% Please scale the page margins according to the number of pages in your document (see optional parameters in the following package "\usepackage{geometry}":

% - up to 199 pages ("inner": 20mm, "outer": 15-18mm) >>> textwidth = 113mm
% - 200 to 399 pages ("inner": 23mm, "outer": 15-18mm) >>> textwidth=110mm
% - from 400 pages ("inner": 25mm, "outer": 15mm) >>> textwidth = 108mm

%%%%%%%%%%%%%%%%%%%%%%%%%%%%%%%%%%%%%%%%%%%%%%%%%%
%%%        			Paper size (17x24)		   %%%
%%%%%%%%%%%%%%%%%%%%%%%%%%%%%%%%%%%%%%%%%%%%%%%%%%

% Note: The margins for scaling (see "Scaling" above) can be managed using the "geometry" package, please refer to the corresponding documentation at: http://ftp.fau.de/ctan/macros/latex/contrib/geometry/geometry.pdf

% (DIN 17x24)
\usepackage[%
papersize={17cm,24cm},% Paper size
headheight=1.5\baselineskip,% Line below header
top=25mm,% Distance from the top edge
inner=20mm,% Distance from the inner edge
outer=15mm,% Distance from the outer edge
%lines=38,% Number of lines. It is recommended to comment out because it leads to problems with the correct calculation
%textwidth=113mm,% Width of the type area. It is recommended to comment out because it leads to problems with the correct calculation
footnotesep=7mm,% Vertical distance between the type area and the footnote line
heightrounded=true%
]{geometry}

%%%%%%%%%%%%%%%%%%%%%%%%%%%%%%%%%%%%%%%%%%%%%%%%%%
%%%        				Footer!				   %%%
%%%%%%%%%%%%%%%%%%%%%%%%%%%%%%%%%%%%%%%%%%%%%%%%%%

% Distance between text body (type area) and bottom edge of footer (page numbers)
\setlength{\footskip}{13.6mm}

% Distance between body text and footnote dividing line
\setlength{\skip\footins}{20pt}

%%%%%%%%%%%%%%%%%%%%%%%%%%%%%%%%%%%%%%%%%%%%%%%%%%
%%%        			Footnote!				   %%%
%%%%%%%%%%%%%%%%%%%%%%%%%%%%%%%%%%%%%%%%%%%%%%%%%%

% (1) Position and size of the footnotes for 2-digit footnote numbers
\deffootnote[1.8em]{1.8em}{0em}{\makebox[1.7em][l]{\textsuperscript{\thefootnotemark\ }}}
% \ deffootnote {"Position of the footnote number from the left (and adjust this value)"} {"Indentation of the second line"} {\ makebox ["Distance between the footnote number and the footnote text (and adjust this value)"] [l] {\ thefootnotemark }}

% (2) Please select the following command for 3-digit footnote numbers and deactivate (1) "\ deffootnote {1.7em} {0em} {\ makebox [1.7em] [l] {\ thefootnotemark}}" by comment out the previous command
%\deffootnote{2.2em}{0em}{\makebox[2.2em][l]{\thefootnotemark}}
% \ deffootnote {"Position of the footnote number from the left (and adjust this value)"} {"Indentation of the second line"} {\ makebox ["Distance between the footnote number and the footnote text (and adjust this value)"] [l] {\ thefootnotemark }}

% Prevents footnotes from continuing on the opposite page
\interfootnotelinepenalty=10000 

%%%%%%%%%%%%%%%%%%%%%%%%%%%%%%%%%%%%%%%%%%%%%%%%%%
%%%  				Paragraph!			  	   %%%
%%%%%%%%%%%%%%%%%%%%%%%%%%%%%%%%%%%%%%%%%%%%%%%%%%

% Distribute lines on the page (the lower margin is not compensated by expanding the paragraph spacing)
\raggedbottom   

% Indentation depth (horizontal distance) of the first line of the paragraph
\setlength{\parindent}{0pt}

% Vertical space between paragraphs
\setlength{\parskip}{2.5mm}

% Prevents orphans (single line at the bottom of the page)
\clubpenalty = 10000 

% prevents widows (single line at the top of the page)
\widowpenalty = 10000
\displaywidowpenalty = 10000

% Prevent hyphenation at the page break
\brokenpenalty = 10000

%%%%%%%%%%%%%%%%%%%%%%%%%%%%%%%%%%%%%%%%%%%%%%%%%%
%%%				Caption! 					   %%%
%%%  			Floating objects			   %%%
%%%				Figure!, Table!			   	   %%%
%%%%%%%%%%%%%%%%%%%%%%%%%%%%%%%%%%%%%%%%%%%%%%%%%%

\usepackage[%
labelfont=bf,% Bold text for the designation "Figure" and "Table"
font=footnotesize,% Font size for captions
]{caption}

% Illustrations
\captionsetup[figure]{position=below}
\captionsetup[figure]{aboveskip=3mm} % Distance between image and caption
\captionsetup[figure]{belowskip=-2mm} % Distance between caption and running text

% Tables
% When using "\captionabove" the values "aboveskip" and "belowskip" are swapped; so please use "\caption" for table headings
\captionsetup[table]{position=top}
\captionsetup[table]{aboveskip=0.8mm} % Spacing: running text - table heading (measured in PDF = ~ 7mm) | Note: with the command "\captionabove" the order of the captions changes to: table heading - table
% \captionsetup[table]{aboveskip=1.5mm} % Spacing: running text - table heading (measured in PDF = ~ 7mm) | Note: with the command "\captionabove" the order of the captions changes to: table heading - table
\captionsetup[table]{belowskip=2.5mm} % Spacing: table heading - table (measured in PDF = ~3mm) | Note: with the command "\captionabove" the order of the captions changes to: running text - table heading

% Intextsep! und textfloat!
% Distance illustration: running text - image; caption - running text
% Space table: running text - table heading; table - running text
\setlength{\intextsep}{5.5mm plus0mm minus0mm}% Distance for floating objects placed with the parameter "[h]" (measured in PDF for tables = ~ 8mm; for images = ~ 6mm)
\setlength{\textfloatsep}{7mm plus0mm minus0mm}% Distance for floating objects placed with the parameter "[t]" or "[b]"

% Caption for subfloat!
\captionsetup[subfloat]{%
	labelformat=empty,%
	margin=0pt,% Indentation of the caption from the left
	% skip = 0pt,% Distance between picture and caption
	aboveskip=2mm, % Distance between picture and caption
	belowskip=0mm,% Distance between the caption and the next row of images as well as between the caption and the signature of the figure / since the commands "\captionsetup[subfloat]{belowskip}" and "\captionskip[figure]{skip}" overlap, the vertical distance between the rows of images must be set with the command "\vspace{3mm}"
	% font = {footnotesize, rm},%
	% labelfont = {footnotesize, bf},%
	% format=hang,% Indent second and further lines (align to the first)
	indention=0em,% Indentation of the inscription
	labelsep=space,% "period, space, quad, newline"
	justification = RaggedRight,% Flutter replacement with hyphenation
	% justification = raggedright,% Flutter replacement without hyphenation
	% justification = centering,% Centered
	% justification = justified,% Justification
	singlelinecheck = true,% "false" (true = always center one line)
	position=auto,% "top, bottom"
	labelformat=parens% "simple, empty" = how the label is set
}

%%%%%%%%%%%%%%%%%%%%%%%%%%%%%%%%%%%%%%%%%%%%%%%%%%
%%%  			Floating! objects			   %%%
%%%				"{figure}", "{table}"	   	   %%%
%%%				Layout, size  	   			   %%%
%%%%%%%%%%%%%%%%%%%%%%%%%%%%%%%%%%%%%%%%%%%%%%%%%%

% Minimum fill level of one side with a floating object
\renewcommand{\floatpagefraction}{0.7}

% Maximum size of a floating object at the bottom of the page
\renewcommand{\topfraction}{0.8}

% Maximum size of a floating object at the top of the page
\renewcommand{\bottomfraction}{0.8}

% Possible increase in spacing within a line if the line break is unsightly
\setlength{\emergencystretch}{4pt}

% Minimum amount of text on a page with a floating object
\renewcommand{\textfraction}{0.1}

%%%%%%%%%%%%%%%%%%%%%%%%%%%%%%%%%%%%%%%%%%%%%%%%%%
%%%  				Line spacing!			   %%%
%%%%%%%%%%%%%%%%%%%%%%%%%%%%%%%%%%%%%%%%%%%%%%%%%%

% Set the line spacing to 1
%\usepackage[singlespacing]{setspace}

% Set line spacing to 1.15
\usepackage{setspace}
\setstretch{1.15}

% Set line spacing to 1.2
%\usepackage{setspace}
%\setstretch{1.2}

% Set line spacing to 1.5
%\usepackage[onehalfspacing]{setspace}

%%%%%%%%%%%%%%%%%%%%%%%%%%%%%%%%%%%%%%%%%%%%%%%%%%
%%%				Directories!				   %%%
%%% 			Table of contents	   		   %%%
%%%				Table of figures 			   %%%
%%%				Table of tables 			   %%%
%%%%%%%%%%%%%%%%%%%%%%%%%%%%%%%%%%%%%%%%%%%%%%%%%%

% toc = table of contents 
% lof = list of figures
% lot = list of tables

\makeatletter
%
% Automatic line break in directories (toc, lof, lot) for long headings, horizontal distance to the right margin of the type area; "plus1fil": No hyphenation in the table of contents, this mostly overrides the justification for directories because hyphenation is not possible
\renewcommand\@tocrmarg{8em plus1fil}%new
% \renewcommand\@tocrmarg{7em plus1fil}%new
% \renewcommand\@tocrmarg{4em plus1fil}%old

% Distance between the page number in the table of contents and the last point of the dotted line
\renewcommand\@pnumwidth{1em} % !With 8pt or 0em, sets the page number outside the type area (see "\usepackage{showframe}"!
%
% Changes the distance between the points of the dotted line
% \renewcommand*{\@dotsep}{1.5}% Default ist 4.5
%
% \renewcommand*\l@chapter{\@dottedtocline{0}{1.5em}{2em}}
%
% \renewcommand*\l@figure{\@dottedtocline{1em}{0em}{2.3em}}% Default is {1.5em}{0em}{2.3em}
% \let\l@table\l@figure
%
\makeatother

% Indentation of the chapter number (tocindent) and space between chapter number and chapter text (heading) in the table of contents (tocnumwidth)
\RedeclareSectionCommand[tocindent=0em,tocnumwidth=1.5em]{chapter}
\RedeclareSectionCommand[tocindent=1.5em,tocnumwidth=1.9em]{section}
\RedeclareSectionCommand[tocindent=3.4em,tocnumwidth=2.6em]{subsection}

% List of figures! Indentation and spacing
\DeclareTOCStyleEntry[%
dynnumwidth=true,% If necessary, the spacing of "numsep" is extended (e.g. with two digits)
indent=0pt,% Indentation left
numsep=1em,% Spacing between the number on the left and the text (title of the illustration), maintains the spacing from the page number
numwidth=3em,% Space between number on the left and text (figure title)
% pagenumberbox=1em,% Space between number on the left and text (figure title)
% pagenumberwidth=1em,% Distance between the page number and the text (left), also shifts the dotted line to the left
% listname={Abbildungsverzeichnis},% Name of the list of figures
% pagenumberformat=\sfamily\large,% Page number font
% entryformat=\sffamily\large,% Directory font
rightindent=8em,% Space between page number on the right and text (figure title), hyphenation is active
]{default}{figure}

% List of tables! Indentation and spacing
\DeclareTOCStyleEntry[%
dynnumwidth=true,% If necessary, the spacing of "numsep" is extended (e.g. with two digits)
indent=0pt,% Indentation of the numbering on the left
numsep=1em,% Space between number on the left and text (table title), keeps the space to page number
numwidth=3em,% Space between number on the left and text (table title)
% pagenumberbox=1em,% Space between number on the left and text (table title)
% pagenumberwidth=1em,% Distance between the page number and the text (left), also shifts the dotted line to the left
% beforeskip=1.15em plus 1pt,%
% linefill=\skillmon@chapter@dotfill,%
% entryformat=\textbf%
% pagenumberbox=\relax,%
% listname={Tabellenverezichnis},% Name of the list of tables
% pagenumberformat=\ssfamily\large,% Page number font
% pagenumberformat=\usekomafont{tocentry},% Alternatively
% entryformat=\sffamily\large,% Directory font
rightindent=8em,% Space between page number on the right and text (table title), hyphenation is active
]{default}{table}

% Directory font family (toc, lof, lot)
% \addtokomafont{disposition}{\sffamily} % Font without serifs
% \addtokomafont{disposition}{\rmfamily} % Font with serifs

% Page numbers in the table of contents for headings in the "\chapter" category
\setkomafont{chapterentrypagenumber}{\rmfamily} % Page numbers in the table of contents with serifs
% \setkomafont{chapterentrypagenumber}{\rmfamily\mdseries} % Page numbers in the table of contents in serifs, but not in bold

% \addtokomafont{chapterentrypagenumber}{\normalfont\normalcolor\fontfamily{phv}\selectfont}

% Font family for headings of the category "\chapter": "\rmfamily" ("\chapter" with serifs); "\sffamily" ("\chapter" without serifs)
% \setkomafont{chapterentry}{\rmfamily\bfseries} 
% \setkomafont{chapterentry}{\sffamily} 

% Standardization of the headings in the table of contents. Changes the display of headings from the "\chapter" category in the table of contents to the "\section" category; Adaptation of the font family of the headings of the category "\chapter": "\rmfamily" (heading of the category "\chapter" with serifs); "\sffamily" (heading of the "\chapter" category without serifs)
% \setkomafont{sectioning}{\rmfamily\normalsize} 
% \setkomafont{sectioning}{\sffamily} 

\KOMAoptions{toc=chapterentrydotfill} % Also add points to the page number in the table of contents for headings in the "\chapter" category

% Show up to level 3 (subsection) in the table of contents
\setcounter{tocdepth}{2} 

% Possibly change in order to get a nicer display of the page
\BeforeStartingTOC{\setstretch{1.075}} 

% Set the font family in the table of contents to sans serif
% \newcommand*\tocentryformat[1]{{\sffamily#1}}
% \RedeclareSectionCommands
%   [
%     tocentryformat=\tocentryformat,
%     tocpagenumberformat=\tocentryformat
%   ]
%   {section,subsection,subsubsection,paragraph,subparagraph}

%%%%%%%%%%%%%%%%%%%%%%%%%%%%%%%%%%%%%%%%%%%%%%%%%%
%%% 		Tocloft! (obsolete)			 	   %%%
%%%%%%%%%%%%%%%%%%%%%%%%%%%%%%%%%%%%%%%%%%%%%%%%%%

% Display the table of contents correctly
% \usepackage[titles]{tocloft}

% Also show points in the chapters
% \renewcommand{\cftchapdotsep}{\cftdotsep}
% \renewcommand{\cftchapleader}{\cftdotfill{\cftchapdotsep}}

% Show page numbers for chapters in sans serif font
% \renewcommand{\cftchappagefont}{\fontfamily{phv}\normalsize\bfseries}

% Update directories
% Fonts in the table of contents
% \renewcommand\cftchapfont{\fontfamily{phv}\normalsize\bfseries}
% \renewcommand\cftsecfont{\fontfamily{phv}\fontsize{11}{11}}

% Fonts in chapters and sections ...
% \renewcommand\cftchappagefont{\fontfamily{phv}\normalsize\bfseries}
% \renewcommand\cftsecpagefont{\fontfamily{phv}\fontsize{11}{11}}

%%%%%%%%%%%%%%%%%%%%%%%%%%%%%%%%%%%%%%%%%%%%%%%%%%
%%% 				Headlines!			  	   %%%
%%%%%%%%%%%%%%%%%%%%%%%%%%%%%%%%%%%%%%%%%%%%%%%%%%

% Number the 4th level (subsubsection)
\setcounter{secnumdepth}{4} 

% Show up to level 3 (subsection) in the table of contents
\setcounter{tocdepth}{2}

% Specify fonts and sizes for the headings
\addtokomafont{chapter}{\fontfamily{phv}\fontsize{20}{22}\bfseries} 		% z. B. "2 State of the art" \fontsize{Font size 18 pt}{space in front of the heading: 20 pt}
\addtokomafont{section}{\fontfamily{phv}\fontsize{15}{17}\bfseries}			% z. B. "2.1 Literature and research" \fontsize{Font size 14 pt}{space in front of the heading: 16 pt}
\addtokomafont{subsection}{\fontfamily{phv}\fontsize{13}{15}\bfseries}		% z. B. "2.1.1 Disciplinary development" \fontsize{Font size 12 pt}{space in front of the heading: 14 pt}
\addtokomafont{subsubsection}{\fontfamily{phv}\fontsize{10}{12}\bfseries}	% z. B. "2.1.1.1 Genesis of scientific concepts" \fontsize{Font size 10 pt}{space in front of the heading: 12 pt}

%%%%%%%%%%%%%%%%%%%%%%%%%%%%%%%%%%%%%%%%%%%%%%%%%%
%%% 				Line up headings 	   	   %%%
%%%%%%%%%%%%%%%%%%%%%%%%%%%%%%%%%%%%%%%%%%%%%%%%%%

% Horizontal distance between numbering and heading
\renewcommand*{\chapterformat}{\makebox[1.4cm][l]{\thechapter\autodot}}
\renewcommand*{\sectionformat}{\makebox[1.4cm][l]{\thesection\autodot}}
\renewcommand*{\subsectionformat}{\makebox[1.4cm][l]{\thesubsection\autodot}}
\renewcommand*{\subsubsectionformat}{\makebox[1.4cm][l]{\thesubsubsection\autodot}}

%%%%%%%%%%%%%%%%%%%%%%%%%%%%%%%%%%%%%%%%%%%%%%%%%%
%%% 		Captions name: figure / table 	   %%%
%%%%%%%%%%%%%%%%%%%%%%%%%%%%%%%%%%%%%%%%%%%%%%%%%%

% Specify captions name for figures
%\addto\captionsngerman{\renewcommand{\figurename}{Abbildung}}
% \addto\captionsenglish{\renewcommand{\figurename}{Abbildung}}
% \newcaptionname{ngerman}\figurename{Abbildung}% 

% Specify captions name for tables
%\addto\captionsngerman{\renewcommand{\tablename}{Tabelle}}
% \addto\captionsenglish{\renewcommand{\tablename}{Tabelle}}
% \newcaptionname{ngerman}\figurename{Abbildung}% 

% Change description / name for listings
\addto\captionsngerman{\renewcommand{\lstlistingname}{\latex source code}}

%%%%%%%%%%%%%%%%%%%%%%%%%%%%%%%%%%%%%%%%%%%%%%%%%%
%%% 		Font size!						   %%%
%%%			headers! and footers!		 	   %%%
%%%		 	page number!, text color!  		   %%%
%%%%%%%%%%%%%%%%%%%%%%%%%%%%%%%%%%%%%%%%%%%%%%%%%%

% Specify the sizes of the captions
%\addtokomafont{caption}{\footnotesize}
%\setkomafont{captionlabel}{\footnotesize}

% Specify the size of the header and footer
\setkomafont{pageheadfoot}{\footnotesize} 

% Size of the page number
\setkomafont{pagenumber}{\normalsize}

% Allow colors in the document
\usepackage{color}
\usepackage{xcolor} % Necessary for the color "gray"

% Define text color black
\color[cmyk]{0,0,0,1}

%%%%%%%%%%%%%%%%%%%%%%%%%%%%%%%%%%%%%%%%%%%%%%%%%%
%%% 				Fonts!					   %%%
%%%%%%%%%%%%%%%%%%%%%%%%%%%%%%%%%%%%%%%%%%%%%%%%%%

%%%%%%%%%%%%%%%%%%%%%%%%%%%%%%%%%%%%%%%%%%%%%%%%%%
%%% 		Fonts with serifs			  	   %%%
%%%%%%%%%%%%%%%%%%%%%%%%%%%%%%%%%%%%%%%%%%%%%%%%%%

% Nimbus 15 Serif
% For example see: https://tug.org/FontCatalogue/nimbus15serif/
\usepackage{nimbusserif}

% URW Nimbus Roman (similar to Times New Roman)
% For example see: https://tug.org/FontCatalogue/urwnimbusroman/ 
%\usepackage{mathptmx}

% Utopia Regular with Fourier
% For example see: https://tug.org/FontCatalogue/utopia-fouriermath/ 
%\usepackage{fourier}

% Utopia Regular with Math Design
% For example see: https://tug.org/FontCatalogue/utopia-mathdesign/ \usepackage[adobe-utopia]{mathdesign}

%%%%%%%%%%%%%%%%%%%%%%%%%%%%%%%%%%%%%%%%%%%%%%%%%%
%%% 		Fonts without serifs		  	   %%%
%%%%%%%%%%%%%%%%%%%%%%%%%%%%%%%%%%%%%%%%%%%%%%%%%%

% Use Helvetica clone (phv) as the standard font for sans serif texts (headings and headings in the table of contents)
\renewcommand{\sfdefault}{phv}

% URW Nimbus Sans
% For example see: https://tug.org/FontCatalogue/urwnimbussans/ 
%\usepackage[scaled]{helvet}
%\renewcommand*\familydefault{\sfdefault}

% Nimbus 15 Sans
% For example see: https://tug.org/FontCatalogue/nimbus15sans/ 
%\usepackage{nimbussans}
%\renewcommand*\familydefault{\sfdefault}

%%%%%%%%%%%%%%%%%%%%%%%%%%%%%%%%%%%%%%%%%%%%%%%%%%
%%%        			Microtype!			   	   %%%
%%%%%%%%%%%%%%%%%%%%%%%%%%%%%%%%%%%%%%%%%%%%%%%%%%
% \usepackage[stretch=10,shrink=10]{microtype} % Prevents blurring and blurring of the font and reduces the number of bad boxes (underfull / overfull); must be included after the font

%%%%%%%%%%%%%%%%%%%%%%%%%%%%%%%%%%%%%%%%%%%%%%%%%%
%%%        		Header! Running title!		   %%%
%%%%%%%%%%%%%%%%%%%%%%%%%%%%%%%%%%%%%%%%%%%%%%%%%%

% Package for the automatic setting of headers and footers
\usepackage[%
markcase=ignoreuppercase,%
automark,%
autooneside=false,%
]{scrlayer-scrpage}
%\usepackage[nouppercase]{scrpage2} % Obsolete package replaced by the package "\usepackage{scrlayer-scrpage}"

% Define the line width in the header
% "The element [headsepline] is applied after \ normalfont and after the elements pageheadfoot and pagehead." (Kohm 2020: 268)
\KOMAoptions{headsepline=0.5pt}
% \setheadsepline{0.5pt}% Command line for the obsolete package "\usepackage{scrpage2}"

% No header / footer on blank pages
%\usepackage{emptypage} % Usage of package `emptypage' together(scrbook) with a KOMA-Script class is not recommended

% Distance between body and line in the header
\setlength{\headsep}{8mm}

%%%%%%%%%%%%%%%%%%%%%%%%%%%%%%%%%%%%%%%%%%%%%%%%%%
%%% 				Notes!		  	   	   	   %%%
%%%%%%%%%%%%%%%%%%%%%%%%%%%%%%%%%%%%%%%%%%%%%%%%%%

% Allows you to insert notes
%\usepackage{todonotes}

% Deactivates all inserted notes
%\usepackage[disable]{todonotes}

% Insert notes in the body text
%\todo[inline]{This is a note.}

%%%%%%%%%%%%%%%%%%%%%%%%%%%%%%%%%%%%%%%%%%%%%%%%%%
%%% 			URLs/links!					   %%%
%%%%%%%%%%%%%%%%%%%%%%%%%%%%%%%%%%%%%%%%%%%%%%%%%%

% Display URLs/links with line breaks
\usepackage[hyphens]{url}

% Line breaks in URLs/links after the following characters
\appto\UrlBreaks{\do\a\do\b\do\c\do\d\do\e\do\f\do\g\do\h\do\i\do\j\do\k\do\l\do\m\do\n\do\o\do\p\do\q\do\r\do\s\do\t\do\u\do\v\do\w\do\x\do\y\do\z\do\/\do\.}

% Settings and display of the URLs/links in the PDF
\usepackage[hidelinks,% Display internet links as normal text
% colorlinks, % Show hyperlinks in color
% citecolor=blue, % Show sources in the running text in the selected font color
pdfpagemode=UseNone,% Do not show bookmarks in PDF reader
pdfpagelayout=TwoColumnRight,% Specify the page display of the PDF document
pdfauthor={\autor},% Author of the PDF document
pdftitle={\pdftitle},% Title of the PDF document
bookmarksnumbered=true,% Numbered chapters also in the PDF navigation
]{hyperref}

%%%%%%%%%%%%%%%%%%%%%%%%%%%%%%%%%%%%%%%%%%%%%%%%%%
%%% 		Settings of other packages		   %%%
%%%%%%%%%%%%%%%%%%%%%%%%%%%%%%%%%%%%%%%%%%%%%%%%%%

% Output of the bibliography
%\bibliographystyle{plaindin}

% Allow embedding of images
\usepackage{graphicx}	

% Enable rotated objects
\usepackage{rotating}

% Extended table environment
\usepackage{tabularx}

% Extended ragged commands
\usepackage{ragged2e}

% Left-justified captions
%\usepackage[justification=RaggedRight]{caption}
%\usepackage[justification=justified]{caption}
%\captionsetup[subfigure]{justification=RaggedRight}

% New column type "L" with width specification for left-justified text
\newcolumntype{L}[1]{>{\RaggedRight\arraybackslash}p{#1}}

%%%%%%%%%%%%%%%%%%%%%%%%%%%%%%%%%%%%%%%%%%%%%%%%%%
%%% 			Mathematics!	 			   %%%
%%%%%%%%%%%%%%%%%%%%%%%%%%%%%%%%%%%%%%%%%%%%%%%%%%

% Mathematical symbols
\usepackage{amsmath}
\usepackage{amssymb}
\usepackage{amsfonts}

% Rows in tables can be joined
\usepackage{multirow}

% Provide additional text symbols
\usepackage{textcomp}

% Define operator symbols
\newcommand{\real}{\operatorname{Re}}				% Real part
\newcommand{\opdiv}{\operatorname{div}}				% Divergence operator
\newcommand{\rot}{\operatorname{rot}}				% Rotation operator
\newcommand{\grad}{\operatorname{grad}}				% Gradient operator
\newcommand{\imag}{\operatorname{Im}}				% Imaginary part
\newcommand{\imein}{\operatorname{j}}				% Imaginary part "j"

% Extended list statements
\usepackage{etoolbox}

% Zero indentation in the list of figures (requires the "tocloft" package)
% \renewcommand{\cftfigindent}{0cm}%obsolete

% Zero indentation in the list of tables (requires the "tocloft" package)
% \renewcommand{\cfttabindent}{0cm}%obsolete
	
% Create an English-language list of figures
\usepackage[english]{nomencl}
%\usepackage[german]{nomencl}

% Rename the command for an entry in the list of abbreviations to "\ sym"
\let\sym\nomenclature

% Change the name of the list of abbreviations
\renewcommand{\nomname}{List of acronyms and symbols}

% Set the column width of the formula characters to "20%" of the text width
\setlength{\nomlabelwidth}{.2\textwidth}

% Include units in the designation and right-justify them
\newcommand{\nomunit}[1]{\renewcommand{\nomentryend}{\hspace*{\fill}#1}}

% Line spacing is reduced to normal text spacing
\setlength\nomitemsep{-\parsep}

% Generate list of abbreviations
\makenomenclature

% Create additional directories
\makeindex

% Change the name of the formula directory
\AtBeginDocument{% 
  %\newcaptionname{ngerman}\equationname{Formel} % 
  %\newcaptionname{ngerman}\listequationname{Formelverzeichnis} % 
  \addtocontents{toc}{\protect\activateonlyattoc} % E.g. for breaks of long headings in the table of contents with the command "\ onlyattoc {\ protect \\}", example: "\ chapter{Long chapter headings and the manual line break \ onlyattoc {\ protect \\} for a clean representation in the text}" (without quotation marks)
}

\DeclareRobustCommand*{\onlyattoc}[1]{} % E.g. for breaks of long headings in the table of contents with the command "\ onlyattoc {\ protect \\}", example: "\ chapter{Long chapter headings and the manual line break \ onlyattoc {\ protect \\} for a clean representation in the text}" (without quotation marks)
\newcommand*{\activateonlyattoc}{\DeclareRobustCommand*{\onlyattoc}[1]{##1}} % E.g. for breaks of long headings in the table of contents with the command "\ onlyattoc {\ protect \\}", example: "\ chapter{Long chapter headings and the manual line break \ onlyattoc {\ protect \\} for a clean representation in the text}" (without quotation marks)

% Formatting for formula directory!
\DeclareNewTOC[% 
  tocentryindent=0pt,%
  tocentrynumwidth=2em,%
  hang=1.5em,%
  type=equation,%
  name={Gl.},% 
  types=equations,% 
  listname={Formelverzeichnis},% 
]{equ} 
\newcommand{\equationentry}[2][\theequation]{ % 
  \addxcontentsline{equ}{equation}[{#1}]{\kern 1em #2} % 
} 
\BeforeStartingTOC[equ]{\def\autodot{:}}

%%%%%%%%%%%%%%%%%%%%%%%%%%%%%%%%%%%%%%%%%%%%%%%%%%
%%%  				Index!					   %%%
%%%%%%%%%%%%%%%%%%%%%%%%%%%%%%%%%%%%%%%%%%%%%%%%%%

% Enables the use of an index or index
\usepackage{makeidx}
% Example: This is an entry \index{entry} in the index.
% Put "\printindex" in the appropriate place in the file "KSP_Diss_17x24.tex"

%%%%%%%%%%%%%%%%%%%%%%%%%%%%%%%%%%%%%%%%%%%%%%%%%%
%%%  				Listings!				   %%%
%%%%%%%%%%%%%%%%%%%%%%%%%%%%%%%%%%%%%%%%%%%%%%%%%%

% Allows the output of LaTeX source code in the text
\usepackage{listings}
% \usepackage{listingsutf8}

% Output umlauts and special characters in listings
\lstset{literate=%
{ä}{{\"a}}1 
{ë}{{\"e}}1 
{ï}{{\"i}}1 
{ö}{{\"o}}1 
{ü}{{\"u}}1
{Ä}{{\"A}}1 
{Ë}{{\"E}}1 
{Ï}{{\"I}}1 
{Ö}{{\"O}}1 
{Ü}{{\"U}}1
{á}{{\'a}}1 
{é}{{\'e}}1 
{í}{{\'i}}1 
{ó}{{\'o}}1 
{ú}{{\'u}}1
{Á}{{\'A}}1 
{É}{{\'E}}1 
{Í}{{\'I}}1 
{Ó}{{\'O}}1 
{Ú}{{\'U}}1
{à}{{\`a}}1 
{è}{{\`e}}1 
{ì}{{\`i}}1 
{ò}{{\`o}}1 
{ù}{{\`u}}1
{À}{{\`A}}1 
{È}{{\'E}}1 
{Ì}{{\`I}}1 
{Ò}{{\`O}}1 
{Ù}{{\`U}}1
{â}{{\^a}}1 
{ê}{{\^e}}1 
{î}{{\^i}}1 
{ô}{{\^o}}1 
{û}{{\^u}}1
{Â}{{\^A}}1 
{Ê}{{\^E}}1 
{Î}{{\^I}}1 
{Ô}{{\^O}}1 
{Û}{{\^U}}1
{œ}{{\oe}}1 
{Œ}{{\OE}}1 
{æ}{{\ae}}1 
{Æ}{{\AE}}1 
{ß}{{\ss}}1
{ű}{{\H{u}}}1 
{Ű}{{\H{U}}}1 
{ő}{{\H{o}}}1 
{Ő}{{\H{O}}}1
{ç}{{\c c}}1 
{Ç}{{\c C}}1
{ã}{{\~a}}1 
{å}{{\r a}}1 
{Å}{{\r A}}1
{ø}{{\o}}1 
{€}{{\EUR}}1 
{£}{{\pounds}}1
{~}{{\textasciitilde}}1
}

% Distance for listings and captions within a listing environment
\lstset{%
aboveskip=5mm,
belowskip=1mm,
abovecaptionskip=0mm,
belowcaptionskip=3mm	
% escapechar=|
}

% Style for listings without line numbers
\lstdefinestyle{kspnonumbers}{%
language=[LaTeX]TeX,
escapechar=|,
commentstyle=\color{gray},
keywordstyle=\color{magenta},
stringstyle=\color{blue},
% backgroundcolor=\color{gray},
% inputpath=folder name
basicstyle=\ttfamily\small\bfseries,
breaklines=true,
showstringspaces=false,
% captionpos=b, % Labeling below the listing
tabsize=2,
frame=tbrl,% Top ("T","t"), buttom ("B", "b"), left ("L", "l"), right ("R", "r")
% frame=single,
framesep=0em,% Width of the frame over the type area margin
framextopmargin=5pt,
framexleftmargin=5pt,
framexrightmargin=5pt,
framexbottommargin=5pt,
% xleftmargin=2em,% Distance from the left edge of the type area inwards
% xrightmargin=2em,% Distance from the right edge of the sentence mirror inwards
% numbers=left,
% numberstyle=\footnotesize\color{gray},
% numbersep=10pt,
% stepnumber=2,
morekeywords={%
chapter
}
}

% % Style for listings with line numbers
% \lstdefinestyle{kspnumbers}{%
% % inputpath=folder name
% basicstyle=\ttfamily\small,
% % backgroundcolor=\color{gray},
% breaklines=true,
% commentstyle=\color{gray},
% keywordstyle=\color{magenta},
% stringstyle=\color{blue}
% % captionpos=b,
% showstringspaces=false,
% numbers=left,
% % numberstyle=\footnotesize\color{gray}
% numbersep=10pt,
% stepnumber=2,
% tabsize=2,
% frame=tblr,% Top, buttom, left, right
% }

%%%%%%%%%%%%%%%%%%%%%%%%%%%%%%%%%%%%%%%%%%%%%%%%%%
%%%  				Latex!					   %%%
%%%%%%%%%%%%%%%%%%%%%%%%%%%%%%%%%%%%%%%%%%%%%%%%%%

% New command for the output of the LaTeX logo
\usepackage{xspace}
\newcommand{\latex}{\LaTeX\xspace}
\newcommand{\tex}{\TeX\xspace}

% Bibliography
%-----------------------
%
% Kohm 2020: Kohm, Markus; Neukamp, Frank; Kielhorn, Axel (2020): Die Anleitung. KOMA-Script. Markus Kohm. 2020-03.12. (Online)

% Title
\newcommand{\headtitle}{Non-Invasive Blood Glucose Monitoring in Ears}

%\makeindex

\newcommand{\nauthor}{Andrej Vladimirovič Ermoshkin}
\newcommand{\akadtitel}{M.Sc.}
\newcommand{\geburtsort}{Geburtsort}

\newcommand{\ntitle}{Non-Invasive Blood Glucose Monitoring in Ears}

\newcommand{\referent}{Prof. Dr.-Ing. Hauptreferent}
\newcommand{\korreferent}{Prof. Dr.-Ing. Korreferent}
\newcommand{\ndatum}{tt.mm.jjjj}
\newcommand{\Erstgutachter}{Eintragen}
\newcommand{\Zweitgutachter}{Eintragen}

%\renewcommand{\figurename}{Abbildung}
%\renewcommand{\tablename}{Tabelle}

%\ohead[]{}
%\ofoot[\pagemark]{\pagemark} % Page numbers at the bottom outside

%\ihead[]{}
%\ohead[]{\headmark} % Headers at the top outside

%%%%%%%%%%%%%%%%%%%%%%%%%%%%%%%%%%%%%%%%%%%%%%%%%%
%%%				Begin document!	             			   %%%
%%%%%%%%%%%%%%%%%%%%%%%%%%%%%%%%%%%%%%%%%%%%%%%%%%

\begin{document}

%%%%%%%%%%%%%%%%%%%%%%%%%%%%%%%%%%%%%%%%%%%%%%%%%%
%%%				Document / Layout			               %%%
%%%%%%%%%%%%%%%%%%%%%%%%%%%%%%%%%%%%%%%%%%%%%%%%%%

% Use the preset page style
% \pagenumbering{gobble}

% Use preset page style for headers or running headers and page numbers (header and footer)
\pagestyle{scrheadings}

% Uses the information from the commands "\markright" or "\markboth" for running headlines or headers (these running headlines are not numbered and the headline has no line)
% \pagestyle{myheadings} 

% Also places the chapter heading on the right side of the header, since there are no subchapters
\renewcommand{\chaptermark}[1]{\markboth{\thechapter\ \  #1}{\thechapter\ \  #1}} 

% Start with 5 (KSP puts 4 labeled pages in front)
%\setcounter{page}{5}

%%%%%%%%%%%%%%%%%%%%%%%%%%%%%%%%%%%%%%%%%%%%%%%%%%
%%%				Title page (include!)		             %%%
%%%%%%%%%%%%%%%%%%%%%%%%%%%%%%%%%%%%%%%%%%%%%%%%%%

\begin{titlepage}
\thispagestyle{empty}

% Hintergrunddeckblatt
\begin{tikzpicture}[remember picture,overlay]
  \node[anchor=north west,inner sep=0pt] at (current page.north west)
    {\includegraphics[width=\paperwidth,height=\paperheight]{logos/KIT_Deckblatt.pdf}};

	% TECO-Logo oben rechts (Abstand von Rand anpassen)
  \node[anchor=north east,inner sep=0pt] at ($(current page.north east)+(-1.5cm,-1.5cm)$)
    {\includegraphics[height=1.2cm]{logos/TECO_KIT.pdf}};
\end{tikzpicture}

% ----- Inhalt zentriert im Rahmen -----
\vspace*{0.5cm} % Abstand von oben, bis Titel startet
\begin{center}
  {\Huge\bfseries \ntitle\par}
  \vspace{1.4cm}
  {\large Seminar Paper\par}
  %\vspace{0.35cm}
  {\normalsize by\par}
  \vspace{0.6cm}
  {\Large\bfseries \nauthor\par}

  \vspace{0.7cm}
  {\normalsize
    Chair of Pervasive Computing Systems/TECO\\
    Institute of Telematics\\
    Department of Informatics\par}

  \vspace{2.0cm}
  \begin{tabular}{@{}p{5.3cm}l@{}}
    First Reviewer:  & \Erstgutachter \\
    Second Reviewer: & \Zweitgutachter \\
    Supervisor:      & Supraja Ramesh \\
  \end{tabular}

  \vspace{1.8cm}
  \normalsize Project Period:\quad 29/04/2025 -- 30/09/2025
\end{center}

\end{titlepage}

%%%%%%%%%%%%%%%%%%%%%%%%%%%%%%%%%%%%%%%%%%%%%%%%%%
%%%			Title part / chapter (include!)	       %%%
%%%%%%%%%%%%%%%%%%%%%%%%%%%%%%%%%%%%%%%%%%%%%%%%%%

% With the command "\frontmatter" of the document class "book", the chapters "Summary", "Foreword", "Table of Contents", "Abbreviations and Symbols" are numbered in small Roman numerals
\frontmatter 

% Summary
\markboth{Abstract}{Abstract}
\chapter*{Abstract}
\label{cha:Abstract}
\addcontentsline{toc}{chapter}{Abstract}

%%%%%%%%%%%%%%%%%%%%%%%%%%%%%%%%%%%%%%%%%%%%%%%%%%

\Blindtext[6]

TODO
test

% Table of contents \ tableofcontents (see table of contents.tex)
\markboth{Contents}{Contents}
%\renewcommand{\contentsname}{Inhaltsverzeichnis}

%\renewcommand\cftdotsep{4.621757}%obsolete  % For larger values, the last point is no longer drawn because it is "outside the line" (requires the "tocloft" package)

%old
% \makeatletter
% \renewcommand{\@tocrmarg}{\@pnumwidth plus1fil} % Further attitude to "points in the table of contents"
% \renewcommand*\@tocrmarg{4em plus1fil} % "4em": Automatic line break in the table of contents with long headings, horizontal distance to the right type area; "plus1fil": No hyphenation in the table of contents
% \makeatother


% \BeforeStartingTOC{\setstretch{1.075}} % Change if necessary to get a nicer page display


\cleardoublepage\pdfbookmark{\contentsname}{toc}\tableofcontents % Name of the table of contents "Table of Contents" in the PDF document (requires the package "\usepackage{bookmark}")

% \renewcommand{\contentsname}{new name} % Changes the heading and bookmark in the PDF file from "Contents" to "new name"; this command must be placed here

%%%%%%%%%%%%%%%%%%%%%%%%%%%%%%%%%%%%%%%%%%%%%%%%%%



% Each new chapter starts on the right side
\cleardoublepage 

%%%%%%%%%%%%%%%%%%%%%%%%%%%%%%%%%%%%%%%%%%%%%%%%%%
%%%				Main parts / Layout			             %%%
%%%%%%%%%%%%%%%%%%%%%%%%%%%%%%%%%%%%%%%%%%%%%%%%%%

% Use Arabic pagination starting on page 1; Due to the document class "book", this is not necessary, since the settings are set with the command "\ mainmatter"
% \pagenumbering{arabic} 
% \setcounter{page}{1}

% With the "\mainmatter" command of the "book" document class, the following chapters are numbered in Arabic numerals. Chapters in the main part are numbered in the table of contents
\mainmatter 

% Line spacing within tables (originally 1.4)
\renewcommand{\arraystretch}{1.2}

% Distance between columns in tables
\setlength{\tabcolsep}{1mm} 

% Spacing for bullets
\setlist[itemize]{itemsep=0mm, topsep=1mm} 

% Greater distance between caption and table
% \captionsetup[table]{belowskip=1mm} 

%\RedeclareSectionCommand[beforeskip=-1.00\baselineskip,afterskip=0.50\baselineskip]{section}
%\RedeclareSectionCommand[beforeskip=-0.75\baselineskip,afterskip=0.50\baselineskip]{subsection}
%\RedeclareSectionCommand[beforeskip=-0.50\baselineskip,afterskip=0.25\baselineskip]{subsubsection}

%%%%%%%%%%%%%%%%%%%%%%%%%%%%%%%%%%%%%%%%%%%%%%%%%%
%%%			Main parts / chapter (include!)	       %%%
%%%%%%%%%%%%%%%%%%%%%%%%%%%%%%%%%%%%%%%%%%%%%%%%%%

% Introduction and Motivation
\markboth{Introduction}{Introduction}

\chapter{Introduction (chapter)}
\label{cha:Introduction}

\section{Motivation}
\label{sec:Motivation}
Diabetes is common. Approximately 37.3 million people in the United States have diabetes, which is about 11\% of the population. Type 2 diabetes is the most common 
form, representing 90\% to 95\% of all diabetes cases. About 537 million adults across the world have diabetes. Experts predict this number will rise to 643 million 
by 2030 and 783 million by 2045.\cite{noauthor_information_2023}

\section{Goal and Scope of This Paper}
\label{sec:Goal and Scope of This Paper}

\markboth{Background: Diabetes and Blood Glucose Monitoring}{Background: Diabetes and Blood Glucose Monitoring}

\chapter{Background: Diabetes and Blood Glucose Monitoring}
\label{cha:Background Diabetes and Blood Glucose Monitoring}

\section{Medical Context}
\label{sec:Medical context}
Diabetes mellitus (I will be referring to it as diabetes but diebetes mellitus is the medical term) is a metabolic disease, involving inappropriately elevated blood 
glucose levels (hyperglycemia).\cite{sapra_diabetes_2023} It can lead to severe complications, such as cardiovascular disease, kidney damage, nerve damage, 
eye and oral complications.\cite{noauthor_diabetes_2025-1} Diabetes can also develop when the body 
of a person isn’t responding to the effects of insulin properly. Diabetes affects people of all ages and most forms of diabetes are chronic.

The most common types of diabetes are Type 2, Prediabetes, Type 1 and Gestational Diabetes. The most common of these is Type 2 Diabetes. This is the type where the body
doesn't respond to insulin poperly or the body doesn't produce enough insulin. It is possible though that both is true for a person. Prediabetes is a condition where 
blood glucose levels are higher than usual, but not as high as to be diagnosed with Type 2 diabetes.\cite{noauthor_information_2023} Type 1 diabetes on the other hand 
is an autoimmune disease, which is a malfunction of the body's immune system that causes the body to attack its own 
tissues.\cite{noauthor_information_2023, fernandez_autoimmune_2024} In this case the immune system attacks insulin producing cells in the pancreas with up to 10\% of 
people having diabetes, having Type 1 diabetes. Another form being Gestational diabetes that develops and usually goes away during pregnancy. But people that had 
Gestational diabetes are at a greater risk of developing Type 2 diabetes later in life.\cite{noauthor_information_2023}

\section{Traditional invasive measurement techniques}
\label{sec:Traditional invasive measurement techniques}
When left unmanaged and untreated, diabetes causes serious health problems, as discribed earlier. So it is crucial to manage diabetes where monitoring blood glucose 
levels is essential. Current/traditional techniques to measure blood glucose levels are invasive. The most common method to measure the blood glucose level is with a
glucose meter, or glucometer. This is a small and portable machine that can measure a person's blood glucose level, requiring only a small sample of blood. There are 
multiple ways to collect the blood sample but the most common one is to prick the finger with a small needle. Other test sites are the upper arm, forearm, base of the 
thumb or the thigh. But readings in the fingertip are much more accurate so preferred. \cite{ellis_blood_2024} 

Another method is continuous glucose monitoring where how
the name already suggests the glucose levels are monitored constantly. The sensor for the monitoring is either inserted under the skin (a small needle) and held in place
with a stick patch (disposable sensor) or it is placed fully under the skin (implantable sensor). These sensors then transmit the data to a reciever, which more often
is a mobile phone. There a person can see its glucose levels, trends and get alarms, if the blood glucose level is too low or high.\cite{wong_continuous_2023}

\section{Need for non-invasive approaches}
\label{sec:Need for non-invasive approaches}
While continuous glucose monitoring (CGM) offers significant advantages over periodic finger-prick testing, such as enabling easier management of blood glucose levels and 
reducing the incidence of acute glycemic emergencies, the invasive nature of traditional methods remains a barrier to widespread and sustained use. Disposable CGM 
sensors must typically be replaced every 7 to 14 days, and implantable variants can last up to 180 days.\cite{wong_continuous_2023}

However, these conventional and minimally invasive methods are often painful, can be costly, and may discourage consistent monitoring, leading to poor adherence to 
testing routines \cite{maldocsda_non-invasive_2024}. By contrast, non-invasive glucose monitoring approaches hold promise for daily and continuous use by being 
painless, more comfortable, and potentially less costly.

Such innovations could significantly improve patient compliance and quality of life, addressing the limitations of traditional monitoring modalities.
\cite{owida_non-invasive_2022, ghosh_evolution_2025} Ultimately, non-invasive technologies may deliver effective, user-friendly alternatives that facilitate 
better long-term management of diabetes.
\markboth{Photoplethysmography}{Photoplethysmography}

\chapter{Photoplethysmography}
\label{cha:Photoplethysmography}

\section{Physical principle}
\label{sec:Physical principle}
Photoplethysmography (PPG) is an optical technique that measures blood-volume changes.
PPG utilizes optical sensors to detect these changes by emitting light into the skin and measuring the amount of light absorbed or reflected by blood vessels.
Depending on the energy of incident photons, bond deformation or vibration at different energy level of different bonds occurs. So, only the photon with energy 
that corresponds to the difference between two of its energy levels can be absorbed.\cite{azudin_principles_2023, castaneda_review_2018}
Glucose absorbs light at the fundamental frequencies (2–2.5 $ \mu $m) and first overtone region (1.53–1.82 $ \mu $m). These wavelengths have the drawbacks of needing 
expensive sensors, strong absorption due to water and scattering of fatty tissue. For the second overtone region (0.8-1.6 $ \mu $m) on the other hand, way cheaper 
infrared sensors can be used. The absorbance of glucose is much weaker compared to the previously mentioned wavelengths, but still detectable compared to other 
tissue chromophores. \cite{hossain_estimation_2019}

\section{Transmission vs. reflection method}
\label{sec:Transmission vs. reflection method}
There are two implementation for PPG:

\textbf{Transmission PPG} places the light source on one side of a thin body part (e.g., fingertip or earlobe) and the photodetector on the opposite side. This way the light traverses the tissue. This method usually yields a higher amplitude and cleaner waveforms. The drawback is that it requires thin, well-perfused bodyparts and is sensitive to local perfusion changes (e.g., cold-induced vasoconstriction).\cite{castaneda_review_2018}
\begin{figure}[H]
    \centering
    \includegraphics[scale=0.5]{Figures/Transmission_ppg.jpg}
    \caption{Transmission PPG on a finger}
    \label{fig:transmission_ppg}
\end{figure}

\textbf{Reflection PPG} places the light source and photodetector on the same side. The light enters, scatters within the tissue, and a portion of it returns to the photodetector. While the amplitude is often lower and more prone to motion artifact than transmission PPG, reflection PPG works on most body sites.\cite{castaneda_review_2018,azudin_principles_2023}
\begin{figure}[H]
    \centering
    \includegraphics[scale=0.75]{Figures/Reflection_ppg.jpg}
    \caption{Reflection PPG principle}
    \label{fig:reflection_ppg}
\end{figure}

\section{Signal characteristics and challenges}
\label{sec:Signal characteristics and challenges}

\textbf{Physiological content.}  
The PPG signal carries a mix of information. The pulsatile part (AC) reflects stroke volume, arterial stiffness, and wave reflections, which can be seen in its amplitude, rise time, and the shape of the dicrotic notch. Breathing introduces slower variations in the baseline. Taking the second derivative of the PPG makes the inflection points more visible and allows the calculation of indices linked to vascular aging and stiffness.\cite{castaneda_review_2018,azudin_principles_2023}

\textbf{Artifacts and confounders.}  
Several factors can distort the signal: Movement of the sensor or tissue can change how light passes through which creates motion artifacts. Stray ambient light may leak into the sensor. Blood flow at the measurement site can also vary with temperature or autonomic activity. For example, vasoconstriction can reduce or even eliminate the signal at extremeties. The amount of pressure from the device itself matters too, since it alters local blood volume. Peripheral sites are the most affected when perfusion is low, while the ear canal has shown stable signals with fewer dropouts, even during induced vasoconstriction or in surgical conditions.\cite{budidha_human_2014,venema_robustness_2014} To improve signal quality, common steps include subtracting ambient light, adaptive filtering, using motion references for regression, robust peak detection, and applying quality checks before extracting features or feeding the data into models.\cite{castaneda_review_2018,azudin_principles_2023}

\textbf{Implications for body-site choice.}  
Where the sensor is placed strongly affects performance. Transmission PPG usually produces higher amplitudes and cleaner signals but often fail during cold-induced vasoconstriction. Reflection PPG setups can be used at most body sites, but they typically have lower amplitude and are more sensitive to motion.\cite{castaneda_review_2018,azudin_principles_2023,budidha_human_2014,venema_robustness_2014}

\markboth{Anatomical zones for measurement}{Anatomical zones for measurement}

\chapter{Anatomical zones for measurement}
\label{cha:Anatomical zones for measurement}

\section{Anatomical zones overview}
\label{sec:Anatomical zones overview}

\section{Advantages and disadvantages of each zone}
\label{sec:Advantages and disadvantages of each zone}

\section{Practical implications for wearable devices}
\label{sec:Practical implications for wearable devices}
\markboth{Classical Machine Learning Models for BGL estimation}{Classical Machine Learning Models for BGL estimation}

\chapter{Classical Machine Learning Models for BGL estimation}
\label{cha:Classical Machine Learning Models for BGL estimation}
\markboth{Deep Learning Approaches}{Deep Learning Approaches}

\chapter{Deep Learning Approaches}
\label{cha:Deep Learning Approaches}

\section{Convolutional Neural Networks (CNNs)}
\label{sec:Convolutional Neural Networks (CNNs)}

\section{Long Short-Term Memory (LSTM) networks}
\label{sec:Long Short-Term Memory (LSTM) networks}

\section{Benefits and challenges}
\label{sec:Benefits and challenges}
\markboth{Hybrid Models}{Hybrid Models}

\chapter{Hybrid Models}
\label{cha:Hybrid Models}

\section{Architecture and workflow}
\label{sec:Architecture and workflow}

\section{Step-by-step process}
\label{sec:Step-by-step process}

\section{Insights from recent literature}
\label{sec:Insights from recent literature}
\markboth{Comparison of Methods}{Comparison of Methods}

\chapter{Comparison of Methods}
\label{cha:Comparison of Methods}

\section{When classical models are preferable}
\label{sec:When classical models are preferable}

\section{When deep/hybrid models outperform}
\label{sec:When deep/hybrid models outperform}

\section{State of the art and emerging trends}
\label{sec:State of the art and emerging trends}

%%%%%%%%%%%%%%%%%%%%%%%%%%%%%%%%%%%%%%%%%%%%%%%%%%
%%%				List of (include!)			             %%%
%%%%%%%%%%%%%%%%%%%%%%%%%%%%%%%%%%%%%%%%%%%%%%%%%%

% List of figures
\listoffigures

% List of tables
\listoftables

% Listings
\renewcommand{\lstlistlistingname}{\latex source code} % Changes the headline and bookmark in the PDF file from "List" to "LaTeX source code"; this command must be placed here
% \lstlistoflistings

% Bibliography
%\markboth{Bibliography}{Bibliography}

%%%%%%%%%%%%%%%%%%%%%%%%%%%%%%%%%%%%%%%%%%%%%%%%%%

\chapter*{List of Publications}
\addcontentsline{toc}{chapter}{List of Publications}

\section*{Journal articles}
\addcontentsline{toc}{section}{Journal articles}
\renewcommand{\refnamejournal}{Journal articles}
\begingroup % Delete heading from \ bibliography {}
\renewcommand{\chapter}[2]{}
\renewcommand{\section}[2]{}
\nocitejournal{*} % Show all titles of the .bib file in the bibliography, if activated; if deactivated, only the quoted titles are displayed
\bibliographystylejournal{plainnat}
\bibliographyjournal{Bibliography/Own_journal_papers}{}
\endgroup

%%%%%%%%%%%%%%%%%%%%%%%%%%%%%%%%%%%%%%%%%%%%%%%%%%

\section*{Conference contributions}
\addcontentsline{toc}{section}{Conference contributions}
\renewcommand{\refnameconference}{Conference contributions}
\begingroup % Delete heading from \ bibliography {}
\renewcommand{\chapter}[2]{}
\renewcommand{\section}[2]{}
\nociteconference{*} % Show all titles of the .bib file in the bibliography, if activated; if deactivated, only the quoted titles are displayed
\bibliographystyleconference{plainnat}
\bibliographyconference{Bibliography/Own_conference_papers}{}
\endgroup

%%%%%%%%%%%%%%%%%%%%%%%%%%%%%%%%%%%%%%%%%%%%%%%%%%

\chapter*{Bibliography}
\addcontentsline{toc}{chapter}{Bibliography}
%\renewcommand{\refname}{Literaturverzeichnis}
%\renewcommand\bibname{Literaturverzeichnis}
\begingroup % Überschrift von \bibliography{} löschen
\renewcommand{\chapter}[2]{}
\renewcommand{\section}[2]{}
\nocite{*} % Show all titles of the .bib file in the bibliography, if activated; if deactivated, only the quoted titles are displayed
\bibliographystyle{plainnat} % Adjust the citation style if necessary
\bibliography{Bibliography/Literature}{}
\endgroup

%%%%%%%%%%%%%%%%%%%%%%%%%%%%%%%%%%%%%%%%%%%%%%%%%%

%\markboth{Literaturverzeichnis}{Literaturverzeichnis}
%\renewcommand{\refname}{Literaturverzeichnis}
%\renewcommand\bibname{Literaturverzeichnis}
%\bibliography{Literatur/Eigene_Konferenz_Papers}{}
%\bibliographystyle{alpha}
%\nocite{*}

\end{document}